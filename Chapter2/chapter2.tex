\chapter{Kiến Thức Nền Tảng}
\ifpdf
    \graphicspath{{Chapter2/Chapter2Figs/PNG/}{Chapter2/Chapter2Figs/PDF/}{Chapter2/Chapter2Figs/}}
\else
    \graphicspath{{Chapter2/Chapter2Figs/EPS/}{Chapter2/Chapter2Figs/}}
\fi

\begin{quote}

Trong chương này, chúng tôi sẽ trình bày những kiến thức nền tảng trên ba chủ đề chính bao gồm\textit{Mạng nơ-ron hồi quy (Recurrent neural network)} và biến thể của nó \textit{Long short-term memory} với khả năng giải quyết vấn đề về các \textit{phụ thuộc dài hạn} (Long term dependencies). Chúng tôi cũng trình bày về mô hình dịch máy nơ-ron dựa trên kiến trúc bộ mã hóa - bộ giải mã đã được đề cập đến trong chương giới thiệu. Những kiến thức được trình bày trong chương này cung cấp những nền tảng cũng như phân tích các vấn đề mà kiến trúc bộ mã hóa - bộ giải mã gặp phải để đi đến chương tiếp theo về cơ chế \textit{Attention} trong dịch máy nơ-ron.

\end{quote}
\section{Mạng nơ-ron hồi quy (Recurrent neural network)}

Trong bài toán dịch máy với dữ liệu văn bản, ta biết rằng những từ trong một câu hoặc một đoạn văn không bao giờ là độc lập với nhau. Ví dụ như trong câu sau \textit{Sư tử là loài động vật ăn \_\_\_}. Dễ biết được rằng từ trong chỗ trống sẽ là \textit{thịt}. Tuy nhiên, trong trường hợp không đọc những từ phía trước, chúng ta không thể nào đoán được từ trong chỗ trống là gì. Điều này có nghĩa là một từ luôn có mối liên hệ với những từ phía trước nó và chúng có một thứ tự. Ta gọi loại dữ liệu có thứ tự là dữ liệu chuỗi.

\textit{Mạng nơ-ron hồi quy (recurrent neural network)} \cite{rnnorigin} gọi tắt là \textit{RNN} là một nhánh của Mạng nơ-ron nhân tạo được thiết kế đặc biệt cho việc mô hình hóa loại dữ liệu chuỗi. Lý do mà mạng nơ-ron hồi quy thích hợp cho dữ liệu chuỗi là vì nó tận dụng được tính chất tuần tự của loại dữ liệu này.

Ta hãy xem xét một ví dụ về não người trong quá trình tạo ra các hành động. Khi chạm tay vào vật nóng, não sẽ tạo ra một tín hiệu để ra lệnh cho cơ thể chúng ta rút tay lại. Không những thế, cảm giác nóng sẽ chuyển hóa thành một suy nghĩ. Suy nghĩ này lại được truyền qua não và dựa trên thông tin của giác quan, ta tiếp tục thực hiện một hành động khác. Ví dụ nhúng tay vào nước khi biết có nước gần đó. Có thể hình dung một cách đơn giản rằng não người hoạt động như một hàm \textit{hồi quy}, nó nhận vào thông tin của giác quan và một suy nghĩ nội tại để tạo ra một hành động và suy nghĩ mới. Suy nghĩ mới này bao gồm cả những suy nghĩ trước đó. Cách hoạt động của não người trong ví dụ trên cũng chính là cách mà mạng nơ-ron hồi quy làm việc. Trong đó, tại mỗi thời điểm, thông tin về giác quan được xem như đầu vào của RNN và đầu ra là một hành động. "Suy nghĩ" hoạt động như một loại bộ nhớ giúp RNN lưu giữ thông tin về bối cảnh, là những gì mà nó đã xử lý trong quá khứ. Việc sở hữu một loại "bộ nhớ" khiến RNN trở thành mô hình phù hợp cho dữ liệu chuỗi. Trong thực tế, mạng nơ-ron hồi quy được áp dụng thành công trong các bài toán mô hình hóa ngôn ngữ (\cite{languagemodelingMikolov1}, \cite{languagemodelingMikolov2},\cite{languagemodelingMikolov3})

%Con người không chỉ thực hiện hành động dựa trên thông tin từ các giác quan hiện tại mà còn chịu ảnh hưởng từ những thông tin mà người đó đã tiếp nhận trước đó. Khi chạm tay vào vật nóng, não người sẽ tạo ra một tín hiệu để rút tay lại. Không những thế, cảm giác nóng sẽ chuyển hóa thành một suy nghĩ. Suy nghĩ này tiếp tục được truyền qua não và dựa trên thông tin của giác quan ta lại tiếp tục thực hiện một hành động khác ví dụ nhúng tay vào nước khi biết có vòi nước gần đó. Có thể hình dung một cách đơn giản rằng não người hoạt động như một hàm \textit{hồi quy}, nó nhận vào thông tin của giác quan và một suy nghĩ nội tại để tạo ra một hành động và một suy nghĩ mới. Suy nghĩ mới này bao gồm cả những suy nghĩ trước đó.

%Trong tự nhiên, có một nhóm dữ liệu được gọi là \textit{dữ liệu chuỗi (sequential data)} ví dụ như lời nói, chuỗi thời gian, dữ liệu cảm biến, video và văn bản,.. Điểm chung của các loại dữ liệu trong nhóm này là tính liên quan tuần tự của chúng: những quan sát phía sau thường có liên quan mật thiết đến những quan sát phía trước. \textit{Mạng nơ-ron hồi quy (recurrent neural network)}\cite{rnnorigin} gọi tắt là \textit{RNN} là một nhánh của Mạng nơ-ron nhân tạo được thiết kế đặc biệt cho việc mô hình hóa loại dữ liệu này.

%Ví dụ như trong câu sau \textit{Sư tử là loài động vật ăn ___}. Dễ biết được rằng từ trong chỗ trống sẽ là \textit{thịt}. Điều này có nghĩa là thông tin về cuối cùng trong câu được 

%Khả năng mô hình hóa dữ liệu chuỗi của RNN có thể được miêu tả như cách hoạt động của não người. Hình dung tại một thời điểm nào đó, não người hoạt động như một hàm máy tính: nó nhận \textit{input} là thông tin của các giác quan và tạo ra \textit{output} dưới dạng hành động (thể hiện ra bên ngoài) và suy nghĩ (nội tại). Lúc còn bé, khi nhìn thấy một con rắn, chúng ta sẽ nghĩ đến "rắn" - một loài vật đáng sợ. Chính suy nghĩ về loài vật này khiến chúng ta nghĩ đến việc "chạy". Rõ ràng "rắn" và "chạy" không phải là những suy nghĩ độc lập mà chúng có một thứ tự. Quan trọng hơn, chính hàm tạo ra suy nghĩ "rắn" từ hình ảnh con rắn cũng biến suy nghĩ "rắn" thành suy nghĩ "chạy". Với ví dụ này, có thể xem não người là một hàm \textit{hồi quy} - được định nghĩa là một hàm tương tự áp dụng cho mọi phần tử của một chuỗi, với đầu ra của nó phụ thuộc vào đầu vào hiện tại và các đầu vào trước đó. Cách hoạt động của mạng nơ-ron hồi quy cũng được mô phỏng dựa trên cơ chế này.

Một mạng nơ-ron hồi quy nhận đầu đầu vào là một chuỗi các vector $x_1, x_2,.., x_n$. Tại thời điểm $t (1 \le t \le n)$ vector $x_t$ thuộc chuỗi đầu vào sẽ được đưa vào mạng nơ-ron hồi quy. RNN xử lý vector đó và cập nhật trạng thái ẩn nội tại của nó được đại diện bởi vector $h_t$. Có thể hình dung $h_t$ như là một bộ nhớ lưu giữ thông tin về các vector mà nó đã xử lý cho đến thời điểm $t$. Ở dạng cơ bản nhất, công thức cập nhật trạng thái ẩn của RNN có dạng:
$$h_t = f \left(x_t, h_{t-1} \right)$$

Trong công thức trên, hàm $f$ tính trạng thái ẩn ở thời điểm $t$ dựa trên đầu vào $x_t$ và trạng thái ẩn tại thời điểm trước đó $h_{t-1}$. Thông thường, hàm $f$ là một hàm phi tuyến như hàm \textit{sigmoid} \cite{sigmoidfunction} hay hàm \textit{tanh} \cite{tanhfunction}.
$$h_t = tanh \left(W_{xh} x_t + W_{hh}h_{t-1} \right)$$

Với $W_{xh}$ và $W_{hh}$ là tham số của mô hình dưới dạng những ma trận số thực. Trạng thái ẩn bắt đầu $h_0$ có thể được khởi tạo bằng 0 hoặc là một vector chứa tri thức có sẵn như trường hợp của bộ giải mã như chúng tôi đã đề cập trong chương 1.




\section{Long short-term memory}

\section{Kiến trúc bộ mã hóa - bộ giải mã}
\subsection{Bộ mã hóa}
\subsection{Bộ giải mã}



