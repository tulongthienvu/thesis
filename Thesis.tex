\documentclass[a4paper,13pt]{extreport}
\usepackage{graphicx}
%\usepackage{indentfirst}
\usepackage{mathptmx}
\usepackage{amsmath,amsfonts,amssymb,fancyhdr}
\usepackage{amsthm,amsxtra,latexsym, amscd}
\usepackage{bm}
\usepackage[utf8]{vietnam}
%\usepackage[unicode,colorlinks=true]{hyperref} 
%\usepackage[utf8]{inputenc}
\usepackage[left=3.50cm, right=2.00cm, top=3.00cm, bottom=3.50cm]{geometry}
\usepackage{tocloft}
\usepackage{type1cm}
\usepackage{vector}
\usepackage{color}
\usepackage[usenames,dvipsnames]{xcolor}
\usepackage{caption/subcaption}
\usepackage[section]{placeins}
%\usepackage{subfigure}
\usepackage{pbox}
%\usepackage[unicode,colorlinks=true]{hyperref}
\usepackage{framed}

\usepackage[hyphens]{url}
%\usepackage[hidelinks]{hyperref}
\usepackage[ unicode, plainpages = false, pdfpagelabels, 
pdfpagelayout = OneColumn, % display single page, advancing flips the page - Sasa Tomic
bookmarks,
bookmarksopen = true,
bookmarksnumbered = true,
breaklinks = true,
linktocpage,
pagebackref,
colorlinks = true,
linkcolor = blue,
urlcolor  = blue,
citecolor = red,
anchorcolor = green,
hyperindex = true,
hyperfigures
]{hyperref} 


%\usepackage{algorithm2e}
\usepackage[chapter]{algorithm}
\usepackage{algpseudocode}
\usepackage{setspace}
\makeatletter
\newcommand{\newalgname}[1]{%
	\renewcommand{\ALG@name}{#1}%
}
\usepackage{longtable}
\usepackage[acronym]{glossaries}
\usepackage{multicol}
\setlength{\columnsep}{25pt}
\usepackage{nomencl}
\makenomenclature
\usepackage{multirow}
\usepackage{pdfpages}

\usepackage{tabularx}
\captionsetup{compatibility=false}
\renewcommand\cftchappresnum{\chaptername\space}
\setlength{\cftchapnumwidth}{2.5cm}
\usepackage{titlesec}
\titleformat{\chapter}[display]
{\normalfont\large\bfseries\centering}{\chaptertitlename\ \thechapter}{20pt}{\huge}
\linespread{1.3}

%\baselineskip 19.5 pt
                 

\begin{document}
	
	%\includepdf[pages=1]{docs/BiaChinh}
	\includepdf{docs/BiaPhu}
	%\includepdf[pages=1-2]{docs/NhanXet}
	
	\pagenumbering{roman}
	\newpage
\chapter*{LỜI CẢM ƠN}
\addcontentsline{toc}{chapter}{LỜI CẢM ƠN}
%\hspace{0.3in}
Trước tiên, em xin gửi lời tri ân sâu sắc đến Thầy Lê Hoài Bắc. Thầy đã rất tận tâm, nhiệt tình hướng dẫn và chỉ bảo em trong suốt quá trình thực hiện luận văn. Không có sự quan tâm, theo dõi chặt chẽ của Thầy chắc chắn em không thể hoàn thành luận văn này.

Em xin chân thành cảm ơn quý Thầy Cô khoa Công Nghệ Thông Tin - trường đại học Khoa Học Tự Nhiên, những người đã ân cần giảng dạy, xây dựng cho em một nền tảng kiến thức vững chắc. 

Con xin cảm ơn ba mẹ đã sinh thành, nuôi dưỡng, và dạy dỗ để con có được thành quả như ngày hôm nay. Ba mẹ luôn là nguồn động viên, nguồn sức mạnh hết sức lớn lao mỗi khi con gặp khó khăn trong cuộc sống.

\hfill TP. Hồ Chí Minh, 3/2014

\hfill \textit{Trần Trung Kiên}

	
	%\newpage
	%\thispagestyle{empty}
	%\mbox{}
	
	%\newpage
	%\addcontentsline{toc}{chapter}{ĐỀ CƯƠNG CHI TIẾT}
	%\includepdf[pages=1-2,pagecommand=\thispagestyle{plain}]{docs/DeCuongChiTiet}
	
	%\newpage
\chapter*{TÓM TẮT}
\addcontentsline{toc}{chapter}{TÓM TẮT} 

 Việc nghiên cứu, xây dựng các thuật toán chất lượng cao để giúp máy tính mô hình hóa, giải thích các hoạt động của con người là một vấn đề ngày càng được quan tâm và đầu tư hơn. Về tổng quan, các mô hình tự động nhận dạng hành động người có tiềm năng ứng dụng rất cao trong thực tế, như: truy vấn video, phân tích hành vi của bệnh nhân trong chẩn đoán bệnh, giám sát an ninh (chống ăn trộm, đánh nhau...), điều khiển video games thông qua cử chỉ và nhiều hệ thống tương tác ngưới-máy khác. Có thể nói, việc đưa ra một giải pháp tổng quát giúp máy hiểu được mọi cử chỉ, hành vi của con người vẫn đang là một bài toán đầy thách thức đối với cộng đồng nghiên cứu, bất chấp những nổ lực rất lớn đã được thực hiện qua hàng thập kỷ.
 
 Sự bùng nổ của công nghệ camera 3 chiều giá thành thấp (như Kinect) trong những năm gần đây đã mở ra nhiều giải pháp giúp đơn giản hóa các tác vụ nhận dạng hành động phức tạp, trong khi vẫn có thể đảm bảo được tiêu chí về tốc độ xử lý thời gian thực. Hòa nhịp cùng với xu hướng nghiên cứu hiện nay, khóa luận tập trung vào giải lớp bài toán nhận dạng hành động người trên dữ liệu đa phương thức RGB-D(gồm thông tin màu và độ sâu) thu được từ Kinect. Trên cơ sở lấy cảm hứng từ các giả thuyết về sự kích thích thị giác ở người tại các vùng nổi bật (\textit{visual attention}), khóa luận tiếp cận theo hướng khai thác các đặc trưng ngữ nghĩa, đặc thù với từng kênh dữ liệu màu-độ sâu, qua đó đề xuất một mô hình kết hợp hiệu quả các thông tin này để biểu diễn và phân lớp các hành động trên video. 

Kết quả sơ bộ mà khóa luận đạt được cho thấy việc trích chọn, kết hợp thông tin đặc trưng từ nhiều kênh dữ liệu như màu-độ sâu là rất cần thiết để tăng cường tri thức cho các hệ thống nhận dạng hành động người. Qua đó, giải pháp trình bày trong khóa luận cũng mở ra một hướng đi hứa hẹn trên con đường giúp máy tính có thể tiến gần hơn tới năng lực nhận thức và cảm thụ thị giác của con người.  
%	Hệ thị giác của người có thể cảm nhận được các cảnh liên tục, nhận biết sự vật và nắm bắt ngữ nghĩa chuyển động dễ dàng. Các nhà thần kinh, tâm lý học đã cố gắng phân tích và giải thích cơ chế giúp hệ thị giác người có thể hoạt động chính xác. Một số lý thuyết / giả thuyết như sự kích thích thị giác tại các vùng nổi bật (\textit{visual attention}), các luật "Gestalt" về cách tổ chức nhận thức đã được đặt ra, giải thích và làm sáng tỏ. Trên cơ sở lấy cảm hứng từ các thuật toán học dựa vào việc phân tích các đặc trưng thị giác nổi bật, chúng tôi cố gắng mô hình hóa và tích hợp các khám phá nhận thức thị giác quan trọng vào một hệ thống nhận dạng cử chỉ tổng quát. Đây cũng là một thành phần cơ bản, quan trọng trong mô hình tổng thể mà chúng ta đang hướng tới – một mô hình chung có thể học và hiểu mọi hoạt động, hành vi của con người.

	
	\newpage
	\renewcommand{\contentsname}{\centerline{MỤC LỤC}}
	\addcontentsline{toc}{chapter}{MỤC LỤC}
	\tableofcontents
	
	\newpage
	\renewcommand\listfigurename{\centerline{DANH MỤC HÌNH ẢNH}}
	\addcontentsline{toc}{chapter}{DANH MỤC HÌNH ẢNH}
	\listoffigures
	
	
	\newpage
	\renewcommand\listtablename{\centerline{DANH MỤC BẢNG}}
	\addcontentsline{toc}{chapter}{DANH MỤC BẢNG}
	\listoftables
	%
	%\newpage
	%\renewcommand{\nomname}{
	%	\addcontentsline{toc}{chapter}{DANH MỤC THUẬT NGỮ VIẾT TẮT}
	%	{\fontsize{25}{25}\selectf>>>>>>> origin/attentionont
	%	DANH MỤC THUẬT NGỮ VIẾT TẮT}
	%	}	
	%%\makeglossaries
	%\nomenclature{$STIP:$}{Space-Time Interest Points - Điểm trọng yếu không-thời gian}
	%\nomenclature{$SC:$}{Sparse Coding}
	%\nomenclature{$BoF:$}{Bag of Feature - Giỏ đặc trưng}
	%\nomenclature{$HOG:$}{Histogram of Oriented Gradients - Lược đồ phân bố cường độ hướng biến thiên độ xám}
	%\nomenclature{$HONV:$}{Histogram of Oriented Normal Vectors - Lược đồ phân bố hướng của các vector pháp tuyến}
	%\nomenclature{$HOF:$}{Histogram of Optical Flows - Lược đồ phân bố 1luồng chuyển động}
	%\nomenclature{$3DS-HONV:$}{3D Spherical-Histogram of Oriented Nomal Vectors}
	%\nomenclature{$SVM:$}{Support Vector Machine - Bô phân lớp "Máy hỗ trợ vector"}
	%\nomenclature{$kNN:$}{k Nearest Neighbors - k láng giềng gần nhất}
	%\nomenclature{$Confusion:$}{Ma trận nhầm lẫn - dùng để mô tả số kết quả phân loại đúng và sai tại mỗi lớp(class)}
	%\nomenclature{$DTW:$}{Dynamic Time Warping - thuật toán so khớp các chuỗi tín hiệu tương tự theo thời gian}
	%\nomenclature{$R.O.I: $}{Region of Interest - Vùng ứng viên hay Vùng quan tâm}
	%\nomenclature{$TGMT: $}{Thị giác máy tính}
	%%\printglossary[type=\acronymtype,title=Abbreviations]
	%\printnomenclature[7em]
	%
	%\newpage
	%%\newglossaryentry{acr_norm}{
	%%  name = $\left\| \right\|$ ,
	%%  description = Phép tính norm,
	%%}
	%\newglossaryentry{acr_concate}{
	%  name = $\odot\hspace{0.2in}$ ,
	%  description = Phép nối 2 vector đặc trưng
	%}
	%\newglossaryentry{acr_early_fusion}{
	%  name = $A-B\hspace{0.2in}$ ,
	%  description = Phép kết hợp trước 2 vector đặc trưng A và B (early fusion)
	%}
	%\newglossaryentry{acr_late_fusion}{
	%  name = $A/B\hspace{0.2in}$ ,
	%  description = Phép kết hợp trễ 2 vector đặc trưng A và B (late fusion)
	%}
	%\newglossaryentry{acr_st}{
	%  name = $s.t.\hspace{0.1in}$ ,
	%  description = thỏa điều kiện (subject to)
	%}
	%\newglossaryentry{acr_hist}{
	%  name = $hist\hspace{0.1in}$ ,
	%  description = Lược đồ phân bố (histogram)
	%}
	%\makeglossaries
	%
	%\printglossary[title=KÝ HIỆU - QUY ƯỚC]{\addcontentsline{toc}{chapter}{KÝ HIỆU-QUY ƯỚC}}
	
	
	%\input{Abstract}
	\newpage
	\pagenumbering{arabic}
	
	% \pagebreak[4]
% \hspace*{1cm}
% \pagebreak[4]
% \hspace*{1cm}
% \pagebreak[4]

\chapter{Giới thiệu }
\ifpdf
    \graphicspath{{Chapter1/Chapter1Figs/PNG/}{Chapter1/Chapter1Figs/PDF/}{Chapter1/Chapter1Figs/}}
\else
    \graphicspath{{Chapter1/Chapter1Figs/EPS/}{Chapter1/Chapter1Figs/}}
\fi

Nhờ vào những cải cách trong giao thông và cơ sở hạ tầng viễn thông mà giờ đây toàn cầu hóa đang trở nên gần với chúng ta hơn bao giờ hết. Trong xu hướng đó nhu cầu giao tiếp và thông hiểu giữa những nền văn hóa là không thể thiếu. Tuy nhiên, những nền văn hóa khác nhau thường kèm theo đó là sự khác biệt về ngôn ngữ, là một trong những trở ngại lớn nhất của sự giao tiếp. Một người phải mất rất nhiều thời gian để thành thạo một ngôn ngữ không phải là tiếng mẹ đẻ, và không thể nào học được nhiều ngôn ngữ cùng lúc. Cho nên, việc phát triển một công cụ để giải quyết vấn đề này là tất yếu. Một trong những công cụ như vậy là \textit{Dịch máy}.

\textit{Dịch máy} là quá trình chuyển đổi văn bản/tiếng nói từ ngôn ngữ này sang dạng tương ứng của nó trong một ngôn ngữ khác, được thực hiện bởi một chương trình máy tính nhằm mục đích cung cấp bản dịch tốt nhất mà không cần sự trợ giúp của con người.

Dịch máy có một quá trình lịch sử lâu dài. Từ thế kỷ XVII, đã có những ý tưởng về việc cơ giới hóa quá trình dịch thuật. Tuy nhiên, đến thế kỷ XX, những nghiên cứu về dịch máy mới thật sự bắt đầu. Vào những năm 1930, Georges Artsrouni người Pháp và Petr Troyanskii người Nga đã nộp bằng sáng chế cho công trình có tên "máy dịch" của riêng họ. Trong số hai người, công trình của Troyanskii có ý nghĩa hơn. Nó đề xuất không chỉ một phương pháp cho bộ từ điển tự động, mà còn là lược đồ cho việc mã hóa các vai trò ngữ pháp song ngữ và một phác thảo về cách phân tích và tổng hợp có thể hoạt động. Tuy nhiên, những ý tưởng của Troyanskii đã không được biết đến cho đến cuối những năm 1950. Trước đó, máy tính đã được phát minh.

Những nỗ lực xây dựng hệ thống dịch máy bắt đầu ngay sau khi máy tính ra đời. Có thể nói, chiến tranh và sự thù địch giữa các quốc gia là động lực lớn nhất cho dịch máy thời bấy giờ. Trong Thế chiến thứ II, máy tính đã được quân đội Anh sử dụng trong việc giải mã các thông điệp được mã hóa của quân Đức. Việc làm này có thể coi là một dạng ẩn dụ của dịch máy khi người ta cố gắng dịch từ tiếng Đức được mã hóa sang tiếng Anh. Trong thời kỳ chiến tranh lạnh, vào tháng 7/1949, Warren Weaver, người được xem là nhà tiên phong trong lĩnh vực dịch máy, đã viết một bản ghi nhớ đưa ra các đề xuất khác nhau của ông trong lĩnh vực này. Những đề xuất đó dựa trên thành công của máy phá mã, sự phát triển của lý thuyết thông tin bởi Claude Shannon và suy đoán về các nguyên tắc phổ quát cơ bản của ngôn ngữ. Trong vòng một năm, một vài nghiên cứu về dịch máy đã bắt đầu tại nhiều trường đại học của Mỹ. Vào ngày 7/1/1954, tại trụ sở chính của IBM ở New York, thử nghiệm Georgetown-IBM được tiến hành. Máy tính IBM 701 đã tự động dịch 49 câu tiếng Nga sang tiếng Anh lần đầu tiên trong lịch sử chỉ sử dụng 250 từ vựng và sáu luật ngữ pháp \cite{hutchins}. Thí nghiệm này được xem như là một thành công và mở ra kỉ nguyên cho những nghiên cứu với kinh phí lớn về dịch máy ở Hoa Kỳ. Ở Liên Xô những thí nghiệm tương tự cũng được thực hiện không lâu sau đó.

Trong một thập kỷ tiếp theo, nhiều nhóm nghiên cứu về dịch máy được thành lập. Một số nhóm chấp nhận phương pháp thử và sai, thường dựa trên thống kê với mục tiêu là một hệ thống dịch máy có thể hoạt động ngay lập tức, tiêu biểu như: nhóm nghiên cứu tại đại học Washington (và sau này là IBM) với hệ thống dịch Nga-Anh cho Không quân Hoa Kỳ, những nghiên cứu tại viện Cơ học Chính xác ở Liên Xô và Phòng thí nghiệm Vật lý Quốc gia ở Anh. Trong khi một số khác hướng đến giải pháp lâu dài với hướng tiếp cận lý thuyết bao gồm cả những vấn đề liên quan đến ngôn ngữ cơ bản như nhóm nghiên cứu tại Trung tâm nghiên cứu lý thuyết tại MIT, Đại học Havard và Đơn vị nghiên cứu ngôn ngữ Đại học Cambridge. Những nghiên cứu trong giai đoạn này có tầm quan trọng và ảnh hưởng lâu dài không chỉ cho Dịch máy mà còn cho nhiều ngành khác như Ngôn ngữ học tính toán, Trí tuệ nhân tạo - cụ thể là việc phát triển các từ điển tự động và kỹ thuật phân tích cú pháp. Nhiều nhóm nghiên cứu đã đóng góp đáng kể cho việc phát triển lý thuyết ngôn ngữ. Tuy nhiên, mục tiêu cơ bản của dịch máy là xây dựng hệ thống có khả năng tạo ra bản dịch tốt lại không đạt được dẫn đến một kết quả là vào năm 1966 bản báo cáo từ Ủy ban tư vấn xử lý ngôn ngữ tự động (Automatic Language Processing Advisory) của Hoa Kỳ, tuyên bố rằng dịch máy là đắt tiền, không chính xác và không mang lại kết quả hứa hẹn \cite{hutchins}. Thay vào đó, họ đề nghị tập trung vào phát triển các từ điển, điều này đã loại bỏ các nhà nghiên cứu Mỹ ra khỏi cuộc đua trong gần một thập kỷ.

\begin{figure}
	\centering
	\includegraphics[width=\textwidth]{mthistory}
	\caption[Lịch sử tóm tắt của dịch máy]{Lịch sử tóm tắt của dịch máy, nguồn ảnh: Ilya Pestov trong blog \href{https://medium.freecodecamp.org/a-history-of-machine-translation-from-the-cold-war-to-deep-learning-f1d335ce8b5}{A history of machine translation from the Cold War to deep learning}}
	\label{fig_mthistory}
\end{figure}

\section{Các phương pháp Dịch máy}

Từ đó đến nay, đã có nhiều hướng tiếp cập đã được sử dụng trong dịch máy với mục tiêu tạo ra bản dịch có độ chính xác cao và giảm thiểu công sức của con người. Trong những năm đầu tiên, để tạo ra bản dịch tốt, các phương pháp thời bấy giờ đều hỏi hỏi những lý thuyết tinh vi về ngôn ngữ học. Hầu hết những hệ thống dịch máy trước những năm 1980 đều là \textit{dịch máy dựa trên luật (Rule-based machine translation - RBMT)}. Những hệ thống này thường bao gồm:
\begin{itemize}
	\item[•] Một từ điển song ngữ (ví dụ từ điển Anh - Đức)
	\item[•] Một tập các luật ngữ pháp (ví dụ trong tiếng Đức, từ kết thúc bằng -heit, -keit, -ung là những từ mang giống cái)		
\end{itemize} 

\begin{figure}
	\centering
	\includegraphics[width=0.5\textwidth]{rulebasedpyramid}
	\caption[Ba phương pháp dịch máy dựa trên luật]{Kim tự tháp của Bernard Vauquois thể hiện ba phương pháp dịch máy dựa luật theo độ sâu của đại diện trung gian. Bắt đầu từ dịch máy trực tiếp đến dịch máy chuyển dịch và trên cùng là dịch máy ngôn ngữ phổ quát (Nguồn: \href{http://en.wikipedia.org/wiki/Machine_translation}{http://en.wikipedia.org/wiki/Machine\_translation})}
	\label{fig_rulebasedpyramid}
\end{figure}

Có ba cách tiếp cận khác nhau theo phương pháp dịch máy dựa trên luật. Bao gồm phương pháp dịch máy trực tiếp, dịch máy chuyển giao và dịch máy ngôn ngữ phổ quát. Mặc dù cả ba đều thuộc về RBMT, tuy nhiên chúng khác nhau về độ sâu của đại diện trung gian. Sự khác biệt này được thể hiện qua kim tự tháp Vauquois, minh họa trên hình \ref{fig_rulebasedpyramid} 
%\begin{itemize}
%	\item[•] \textit{Dịch máy trực tiếp} (Direct machine translation - DMT): Đây là phương pháp đơn giản nhất của dịch máy. DMT không dùng bất cứ dạng đại diện nào của ngôn ngữ nguồn, nó chia câu thành các từ, dịch chúng bằng một từ điển song ngữ. Sau đó, dựa trên các luật mà những nhà ngôn ngữ học đã xây dựng, nó chỉnh sửa để bản dịch trở nên đúng cú pháp và ít nhiều đúng về mặt phát âm.
%	\item[•] \textit{Dịch máy ngôn ngữ phổ quát} (Interlingual machine translation - IMT): Trong phương pháp này, câu nguồn được chuyển thành biểu diễn trung gian và biểu diễn này được thống nhất cho tất cả ngôn ngữ trên thế giới (interlingua). Tiếp theo, dạng đại diện này sẽ được chuyển đổi sang bất kỳ ngôn ngữ đích nào. Một trong những ưu điểm chính của hệ thống này là tính mở rộng của nó khi số lượng ngôn ngữ cần dịch tăng lên. Mặc dù trên lý thuyết, phương pháp này trông rất hoàn hảo. Nhưng trong thực tế, thật khó để tạo được một ngôn ngữ phổ quát như vậy.
%	\item[•] \textit{Dịch máy chuyển giao} (Transfer-based machine translation - TMT): dịch máy chuyển giao tương tự như dịch máy ngôn ngữ đại diện ở chỗ, nó cũng tạo ra bản dịch từ biểu diễn trung gian mô phỏng ý nghĩa của câu gốc. Không giống như IMT, TMT phụ thuộc một phần vào cặp ngôn ngữ mà nó tham gia vào quá trình dịch. Trên cơ sở sự khác biệt về cấu trúc của ngôn ngữ nguồn và ngôn ngữ đích, một hệ thống TMT có thể được chia thành ba giai đoạn: i) Phân tích, ii) Chuyển giao, iii) Tạo ra bản dịch. Trong giai đoạn đầu tiên, trình phân tích cú pháp ở ngôn ngữ nguồn được sử dụng để tạo ra biểu diễn cú pháp của câu nguồn. Trong giai đoạn tiếp theo, kết quả của phân tích cú pháp được chuyển đổi thành biểu diễn tương đương trong ngôn ngữ đích. Trong giai đoạn cuối cùng, một bộ phân tích hình thái của ngôn ngữ đích được sử dụng để tạo ra các bản dịch cuối cùng.
%\end{itemize}

\textit{Dịch máy trực tiếp} (Direct machine translation - DMT): Đây là phương pháp đơn giản nhất của dịch máy. DMT không dùng bất cứ dạng đại diện nào của ngôn ngữ nguồn, nó chia câu thành các từ, dịch chúng bằng một từ điển song ngữ. Sau đó, dựa trên các luật mà những nhà ngôn ngữ học đã xây dựng, nó chỉnh sửa để bản dịch trở nên đúng cú pháp và ít nhiều đúng về mặt phát âm.

\textit{Dịch máy ngôn ngữ phổ quát} (Interlingual machine translation - IMT): Trong phương pháp này, câu nguồn được chuyển thành biểu diễn trung gian và biểu diễn này được thống nhất cho tất cả ngôn ngữ trên thế giới (interlingua). Tiếp theo, dạng đại diện này sẽ được chuyển đổi sang bất kỳ ngôn ngữ đích nào. Một trong những ưu điểm chính của hệ thống này là tính mở rộng của nó khi số lượng ngôn ngữ cần dịch tăng lên. Mặc dù trên lý thuyết, phương pháp này trông rất hoàn hảo. Nhưng trong thực tế, thật khó để tạo được một ngôn ngữ phổ quát như vậy.

\textit{Dịch máy chuyển giao} (Transfer-based machine translation - TMT): dịch máy chuyển giao tương tự như dịch máy ngôn ngữ đại diện ở chỗ, nó cũng tạo ra bản dịch từ biểu diễn trung gian mô phỏng ý nghĩa của câu gốc. Tuy nhiên, không giống như IMT, TMT phụ thuộc một phần vào cặp ngôn ngữ mà nó tham gia vào quá trình dịch. Trên cơ sở sự khác biệt về cấu trúc của ngôn ngữ nguồn và ngôn ngữ đích, một hệ thống TMT có thể được chia thành ba giai đoạn: i) Phân tích, ii) Chuyển giao, iii) Tạo ra bản dịch. Trong giai đoạn đầu tiên, trình phân tích cú pháp ở ngôn ngữ nguồn được sử dụng để tạo ra biểu diễn cú pháp của câu nguồn. Trong giai đoạn tiếp theo, kết quả của phân tích cú pháp được chuyển đổi thành biểu diễn tương đương trong ngôn ngữ đích. Trong giai đoạn cuối cùng, một bộ phân tích hình thái của ngôn ngữ đích được sử dụng để tạo ra các bản dịch cuối cùng.

Mặc dù đã có một số hệ thống RBMT được đưa vào sử dụng như PROMPT \cite{promt} và Systrans \cite{systrans}. Tuy nhiên, bản dịch của hướng tiếp cận này có chất lượng thấp so với nhu cầu của con người và không sử dụng được trừ một số trường hợp đặc biệt. Ngoài ra chúng còn có một số nhược điểm lớn như:
\begin{itemize}
	\item[•] Các loại từ điển chất lượng tốt có sẵn là không nhiều và việc xây dựng những bộ từ điển mới là rất tốn kém.
	\item[•] Hầu hết những luật ngôn ngữ được tạo ra bằng tay bởi các nhà ngôn ngữ học. Việc này gây khó khăn và tốn kém khi hệ thống trở nên lớn hơn.
	\item[•] Các hệ thống RBMT gặp khó khăn trong việc giải quyết những vấn đề như thành ngữ hay sự nhập nhằng về ngữ nghĩa của các từ. 
\end{itemize}
Từ những năm 1980, dịch máy dựa trên \textit{Ngữ liệu} (Corpus-based machine translation) được đề xuất. Điểm khác biệt lớn nhất và cũng là quan trọng nhất của hướng tiếp cận này so với RBMT là thay vì sử dụng các bộ từ điển song ngữ, nó dùng những tập câu tương đương trong hai ngôn ngữ làm nền tảng cho việc dịch thuật. Tập những câu tương đương này được gọi là ngữ liệu. So với từ điển, việc thu thập ngữ liệu đơn giản hơn rất nhiều. Ví dụ như ta có thể tìm thấy nhiều phiên bản trong các ngôn ngữ khác nhau của những văn bản hành chính hay các trang web đa ngôn ngữ. Trước khi dịch máy nơ-ron ra đời, dịch máy dựa trên ngữ liệu bao gồm hai phương pháp: dịch máy dựa trên ví dụ và dịch máy thống kê.

%Nhóm thứ hai là những hướng tiếp cận dựa trên \textit{Ngữ liệu} (Corpus based). Nhóm này hoạt động dựa trên một tập dữ liệu song song của các cặp câu là bản dịch của nhau trong hai ngôn ngữ gọi là ngữ liệu và chỉ yêu cầu những tri thức tối thiểu về ngôn ngữ học. Trước khi dịch máy nơ-ron ra đời, phương pháp nổi bật và hiệu quả nhất dựa trên hướng tiếp cận này chính là \textit{Dịch máy thống kê} (Statistical machine translation). Vào năm 1990, IBM công bố hệ thống dịch máy thống kê của họ, đây là hệ thống đầu tiên có khả năng tạo ra bản dịch mà không cần biết gì về các từ hay quy tắc ngữ pháp của ngôn ngữ. Chỉ cần cung cấp bộ ngữ liệu, hệ thống này phân tích các câu tương ứng trong ngữ liệu đó để hiểu được các mô hình bên dưới.
\textit{Dịch máy dựa trên ví dụ} (Example-based Machine Translation - EBMT): 

\textit{Dịch máy thống kê} (Statistical machine translation - SMT): ý tưởng của phương pháp này là thay vì định nghĩa những từ điển và các luật ngữ pháp một cách thủ công, SMT dùng mô hình thống kê để học các từ điển và các luật ngữ pháp này từ ngữ liệu. Những ý tưởng đầu tiên của SMT được giới thiệu đầu tiên bởi Waren Weaver vào năm 1949 bap gồm việc áp dụng lý thuyết thông tin của Claude Shannon vào dịch máy. SMT được giới thiệu lại vào cuối những năm 1980 và đầu những năm 1990 tại trung tâm nghiên cứu Thomas J. Watson của IBM. SMT là phương pháp được nghiên cứu rộng rãi nhất thời bấy giờ và thậm chí đến hiện tại, nó vẫn là một trong những phương pháp được nghiên cứu nhiều nhất về dịch máy. 

Để hiểu rõ hơn về dịch máy thống kê, xét một ví dụ: ta cần dịch một câu $f$ trong tiếng Pháp sang dạng tiếng Anh $e$ của nó. Có nhiều bản dịch có thể có cuả $f$ trong tiếng Anh, việc cần làm là chọn $e$ sao cho nó là bản dịch "tốt nhất" của $f$. Chúng ta có thể mô hình hóa quá trình này bằng một xác suất có điều kiện $p(e|f)$ với $e$ là những bản dịch có thể có với câu cho trước $f$. Một cách hợp lý để chọn bản dịch "tốt nhất" là chọn $e$ sao cho nó tối đa xác suất có điều kiện $p(e|f)$. Cách tiếp cận quen thuộc là sử dụng định lý Bayes để viết lại $p(e|f)$:
\begin{equation} \label{bayesFomular}
	p(e|f) = \frac{p(f|e)p(e)}{p(f)}
\end{equation}
Bởi vì $f$ là cố định, tối đa hóa $p(e|f)$ tương đương với tìm $e$ sao cho tối đa hóa $p(f|e)p(e)$. Để làm được điều này, chúng ta dựa vào một tập ngữ liệu là những câu song ngữ Anh - Pháp để suy ra các mô hình $p(f|e)$ và $p(e)$ và sử dụng những mô hình đó để tìm một bản dịch cụ thể $\tilde{e}$ sao cho:
\begin{equation} \label{ehatSMT}
	\tilde{e} = \arg\max_{e \in e^*} p(e|f) = \arg\max_{e \in e^*} p(f|e)p(e)
\end{equation}
Ở đây, $p(f|e)$ được gọi là \textit{mô hình dịch} (translation model) và $p(e)$ được gọi là \textit{mô hình ngôn ngữ} (language model). Mô hình dịch $p(f|e)$ thể hiện khả năng câu $e$ là một bản dịch của câu $f$. Những mô hình dịch ban đầu dựa trên từ (word-based) như các mô hình IBM 1-5 (IBM Models 1-5). Những năm 2000, những mô hình dịch dựa trên cụm từ (phrase based) xuất hiện giúp cải thiện khả năng dịch của SMT. Trong khi đó, mô hình ngôn ngữ $p(e)$ thể hiện độ trơn tru của câu $e$. Ví dụ $p($"tôi đi học"$) >$ $p($"học tôi đi"$)$ vì rõ ràng "tôi đi học" là có lý hơn "học tôi đi". Các mô hình ngôn ngữ cho SMT thường được ước lượng bằng các mô hình n-gram được làm mịn, cách làm này cũng là một nhược điểm của SMT. Mô hình ngôn ngữ là một chủ đề quan trọng và sẽ được chúng tôi đề cập lại một lần nữa trong chương Kiến thức nền tảng.

\begin{figure}
	\centering
	\includegraphics[width=\textwidth]{smt}
	\caption[Ví dụ về tập các câu song song trong hai ngôn ngữ]{Ví dụ về tập các câu song song trong hai ngôn ngữ}
	\label{fig_parallelcorpus}
\end{figure}


\section{Dịch máy Nơ-ron}

Mặc dù trên thực tế đã có nhiều hệ thống dịch máy được phát triển dựa trên dịch máy thống kê thời bấy giờ, tuy nhiên nó không hoạt động thực sự tốt bởi một số nguyên nhân. Một là việc những từ hay đoạn được dịch cục bộ và quan hệ của chúng với những từ cách xa trong câu nguồn thường bị bỏ qua. Hai là mô hình ngôn ngữ N-gram hoạt động không thực sự tốt đối với những bản dịch dài và ta phải tốn nhiều bộ nhớ để lưu trữ chúng. Ngoài ra việc sử dụng nhiều thành phần nhỏ được điều chỉnh riêng biệt như mô hình dịch, mô hình ngôn ngữ,.. cũng gây khó khăn cho việc vận hành và phát triển mô hình này.

% TODO: Kalchbrenner and Blunsom (2013), Sutskever et al. (2014) and Cho et al. (2014b)
\textit{Dịch máy nơ-ron} (Neural machine translation) là một hướng tiếp cận mới trong dịch máy trong những năm gần đây được đề xuất đầu tiên bởi \cite{kalchbrennerBlunsom}, \cite{sutskever}, \cite{cho}. Giống như dịch máy thống kê, dịch máy nơ-ron cũng là một phương pháp thuộc hướng tiếp cận dựa trên ngữ liệu, trong khi dịch máy thống kê bao gồm nhiều mô-đun nhỏ được điều chỉnh riêng biệt, Dịch máy nơ-ron cố gắng dùng một mạng nơ-ron như là thành phần duy nhất của hệ thống, mọi thiết lập sẽ được thực hiện trên mạng này. 

% TODO: Sutskever et al., 2014; Cho et al., 2014a
Hầu hết những mô hình dịch máy nơ-ron đều dựa trên kiến trúc \textit{Bộ mã hóa - Bộ giải mã} (Encoder-Decoder) (\cite{sutskever}, \cite{cho}). Bộ mã hóa thường là một mạng nơ-ron có tác dụng \textit{"nén"} tất cả thông tin của câu trong ngôn ngữ nguồn vào một vector có kích thước cố định. Bộ giải mã, cũng là một mạng nơ-ron, sẽ tạo bản dịch trong ngôn ngữ đích từ vector có kích thước cố định kia. Toàn bộ hệ thống bao gồm bộ mã hóa và bộ giải mã sẽ được huấn luyện \textit{"end-to-end"} để tạo ra bản dịch, quá trình này được mô tả như hình 1.2.

\begin{figure}
	\centering
	\includegraphics[width=\textwidth]{intro2nmt}
	\caption[Ví dụ về Kiến trúc \textit{bộ mã hóa - bộ giải mã} trong dịch máy nơ-ron]{Ví dụ về kiến trúc bộ mã hóa - bộ giải mã trong dịch máy nơ-ron}
	\label{fig_encoder_decoder}
\end{figure}

% TODO: mention LSTM
Trong thực tế cả bộ mã hóa và giải mã thường dựa trên một mô hình mạng nơ-ron tên là \textit{Mạng nơ-ron hồi quy} là một thiết kế mạng đặc trưng cho việc xử lý dữ liệu chuỗi. Mạng nơ-ron hồi quy cho phép chúng ta mô hình hóa những dữ liệu có độ dài không xác định, rất thích hợp cho bài toán dịch máy. Hình 1.3 mô tả chi tiết hơn về kiến trúc bộ mã hóa - giải mã sử dụng mạng nơ-ron hồi quy. Đầu tiên bộ mã hóa đọc qua toàn bộ câu nguồn và tạo ra một vector đại diện gọi là \textit{vector trạng thái}. Điều này giúp cho toàn bộ những thông tin cần thiết hay quan hệ giữa các từ đều được tập hợp vào một nơi duy nhất. Bộ giải mã, lúc này đóng vai trò như một mô hình ngôn ngữ để tạo ra từng từ trong ngôn ngữ đích và sẽ dừng lại đến khi một ký tự đặc biệt xuất hiện.

\begin{figure}
	\centering
	\includegraphics[width=\textwidth]{encoder-decoder}
	\caption[Kiến trúc bộ mã hóa - bộ giải mã được xây dựng trên mạng nơ-ron hồi quy]{Kiến trúc bộ mã hóa - bộ giải mã được xây dựng trên mạng nơ-ron hồi quy}
	\label{fig_encoder_decoder_details}
\end{figure}

Trong hình 2, có thể thấy rằng bộ giải mã tạo ra bản dịch chỉ dựa trên trạng thái ẩn cuối cùng, cũng chính là vector có kích thước cố định được tạo ra ở bộ mã hóa. Vector này phải mã hóa mọi thứ chúng ta cần biết về câu nguồn. Giả sử chúng ta có câu nguồn với độ dài là 50 từ, từ đầu tiên ở câu đích có lẽ sẽ có mối tương quan cao với từ đầu tiên ở câu nguồn. Điều này có nghĩa là bộ giải mã phải xem xét thông tin được mã hóa từ 50 \textit{"time step"} trước đó. Mạng nơ-ron hồi quy được chứng minh là gặp khó khăn trong việc mã hóa những chuỗi dài \cite{difficultyRNN}. Để giải quyết vấn đề này, thay vì dùng mạng nơ-ron hồi quy thuần, người ta sử dụng các biến thể của nó quy như \textit{Long short-term memory (LSTM)}. Trên lý thuyết, LSTM có thể giải quyết vấn đề mất mát thông tin trong chuỗi dài, nhưng trong thực tế vấn đề này vẫn chưa thể được giải quyết hoàn toàn. Một số nhà nghiên cứu đã phát hiện ra rằng đảo ngược chuỗi nguồn trước khi đưa vào bộ mã hóa tạo ra kết quả tốt hơn một cách đáng kể \cite{sutskever} bởi nó khiến cho những từ đầu tiên được đưa vào bộ mã hóa sau cùng, và được giải mã thành từ tương ứng ngay sau đó. Cách làm này tuy giúp cho bản dịch hoạt động tốt hơn trong thực tế, nhưng nó không phải là một giải pháp về mặt thuật toán. Hầu hết các đánh giá về dịch máy được thực hiện trên các ngôn ngữ như ngôn ngữ có trật tự câu tương đối giống nhau. Ví dụ trật tự dạng "chủ ngữ - động từ - vị ngữ" như tiếng Anh, Đức, Pháp hay Trung Quốc. Đối với dạng ngôn ngữ có một trật tự khác ví dụ "chủ ngữ - vị ngữ - động từ" như tiếng Nhật, đảo ngược câu nguồn sẽ không hiệu quả.

\textit{Attention} là cơ chế giải phóng kiến trúc bộ mã hóa - bộ giải mã khỏi nhược điểm chỉ sử dụng một vector có chiều dài cố định làm đại diện cho câu đầu vào. Ý tưởng chính của cơ chế này là ở mỗi thời điểm phát sinh các từ trong bản dịch, bộ giải mã sẽ "nhìn" vào các phần khác nhau của câu nguồn trong quá trình mã hóa. Quan trọng hơn, cơ chế này cho phép mô hình học được cách chọn những phần cần thiết để tập trung vào dựa trên câu nguồn và những gì mà bộ giải mã đã giải mã được.

\begin{figure}
	\centering
	\includegraphics[width=\textwidth]{intro2attention}
	\caption[Cơ chế Attention trong dịch máy nơ-ron]{Cơ chế Attention trong dịch máy nơ-ron}
	\label{fig_introattention}
\end{figure} 

\section{Cấu trúc của khóa luận}
Trong khóa luận này, chúng tôi quyết định tập trung nghiên cứu về dịch máy nơ-ron và cơ chế Attention dựa trên nghiên cứu của nhóm tác giả tại đại học Stanford bao gồm Minh-Thang Luong, Hieu Pham, Christopher Manning trong bài báo \textit{Effective Approaches to Attention-based Neural Machine Translation} \cite{mainpaper}. Các phần còn lại trong luận văn được trình bày như sau:

\begin{itemize}
	\item[•] Chương 2 trình bày về những thành nền tảng của kiến trúc bộ mã hóa - giải mã
	
	\item[•] Chương 3 trình bày về cơ chế Attention, đây là phần chính của luận văn. Trong phần này gồm có hai phần nhỏ:
		\begin{itemize}
			\item[-] \textit{Global attetion}: là cơ chế tập trung vào tất cả các trạng thái ở câu nguồn
			\item[-] \textit{Local attetion}: tập trung vào một tập các trạng thái ở câu nguồn tại một thời điểm
		\end{itemize}
	\item[•] Chương 4 trình bày về các thí nghiệm và các phân tích về kết quả đạt trên hai tập dữ liệu Anh-Đức, Anh-Việt.
	\item[•] Kết luận và hướng phát triển của luận văn.
\end{itemize}






	\chapter{Kiến Thức Nền Tảng}
\ifpdf
    \graphicspath{{Chapter2/Chapter2Figs/PNG/}{Chapter2/Chapter2Figs/PDF/}{Chapter2/Chapter2Figs/}}
\else
    \graphicspath{{Chapter2/Chapter2Figs/EPS/}{Chapter2/Chapter2Figs/}}
\fi

\begin{quote}

Trong chương này, chúng tôi sẽ trình bày những kiến thức nền tảng trên ba chủ đề chính bao gồm\textit{Mạng nơ-ron hồi quy (Recurrent neural network)} và biến thể của nó \textit{Long short-term memory} với khả năng giải quyết vấn đề về các \textit{phụ thuộc dài hạn} (Long term dependencies). Chúng tôi cũng trình bày về mô hình dịch máy nơ-ron dựa trên kiến trúc bộ mã hóa - bộ giải mã đã được đề cập đến trong chương giới thiệu. Những kiến thức được trình bày trong chương này cung cấp những nền tảng cũng như phân tích các vấn đề mà kiến trúc bộ mã hóa - bộ giải mã gặp phải để đi đến chương tiếp theo về cơ chế \textit{Attention} trong dịch máy nơ-ron.

\end{quote}
\section{Mạng nơ-ron hồi quy (Recurrent neural network)}

Trong bài toán dịch máy với dữ liệu văn bản, ta biết rằng những từ trong một câu hoặc một đoạn văn không bao giờ là độc lập với nhau. Ví dụ như trong câu sau \textit{Sư tử là loài động vật ăn \_\_\_}. Dễ biết được rằng từ trong chỗ trống sẽ là \textit{thịt}. Tuy nhiên, trong trường hợp không đọc những từ phía trước, chúng ta không thể nào đoán được từ trong chỗ trống là gì. Điều này có nghĩa là một từ luôn có mối liên hệ với những từ phía trước nó và chúng có một thứ tự. Ta gọi loại dữ liệu có thứ tự là dữ liệu chuỗi.

\textit{Mạng nơ-ron hồi quy (recurrent neural network)} \cite{rnnorigin} gọi tắt là \textit{RNN} là một nhánh của Mạng nơ-ron nhân tạo được thiết kế đặc biệt cho việc mô hình hóa loại dữ liệu chuỗi. Lý do mà mạng nơ-ron hồi quy thích hợp cho dữ liệu chuỗi là vì nó tận dụng được tính chất tuần tự của loại dữ liệu này.

Ta hãy xem xét một ví dụ về não người trong quá trình tạo ra các hành động. Khi chạm tay vào vật nóng, não sẽ tạo ra một tín hiệu để ra lệnh cho cơ thể chúng ta rút tay lại. Không những thế, cảm giác nóng sẽ chuyển hóa thành một suy nghĩ. Suy nghĩ này lại được truyền qua não và dựa trên thông tin của giác quan, ta tiếp tục thực hiện một hành động khác. Ví dụ nhúng tay vào nước khi biết có nước gần đó. Có thể hình dung một cách đơn giản rằng não người hoạt động như một hàm \textit{hồi quy}, nó nhận vào thông tin của giác quan và một suy nghĩ nội tại để tạo ra một hành động và suy nghĩ mới. Suy nghĩ mới này bao gồm cả những suy nghĩ trước đó. Cách hoạt động của não người trong ví dụ trên cũng chính là cách mà mạng nơ-ron hồi quy làm việc. Trong đó, tại mỗi thời điểm, thông tin về giác quan được xem như đầu vào của RNN và đầu ra là một hành động. "Suy nghĩ" hoạt động như một loại bộ nhớ giúp RNN lưu giữ thông tin về bối cảnh, là những gì mà nó đã xử lý trong quá khứ. Việc sở hữu một loại "bộ nhớ" khiến RNN trở thành mô hình phù hợp cho dữ liệu chuỗi. Trong thực tế, mạng nơ-ron hồi quy được áp dụng thành công trong các bài toán mô hình hóa ngôn ngữ (\cite{languagemodelingMikolov1}, \cite{languagemodelingMikolov2},\cite{languagemodelingMikolov3})

%Con người không chỉ thực hiện hành động dựa trên thông tin từ các giác quan hiện tại mà còn chịu ảnh hưởng từ những thông tin mà người đó đã tiếp nhận trước đó. Khi chạm tay vào vật nóng, não người sẽ tạo ra một tín hiệu để rút tay lại. Không những thế, cảm giác nóng sẽ chuyển hóa thành một suy nghĩ. Suy nghĩ này tiếp tục được truyền qua não và dựa trên thông tin của giác quan ta lại tiếp tục thực hiện một hành động khác ví dụ nhúng tay vào nước khi biết có vòi nước gần đó. Có thể hình dung một cách đơn giản rằng não người hoạt động như một hàm \textit{hồi quy}, nó nhận vào thông tin của giác quan và một suy nghĩ nội tại để tạo ra một hành động và một suy nghĩ mới. Suy nghĩ mới này bao gồm cả những suy nghĩ trước đó.

%Trong tự nhiên, có một nhóm dữ liệu được gọi là \textit{dữ liệu chuỗi (sequential data)} ví dụ như lời nói, chuỗi thời gian, dữ liệu cảm biến, video và văn bản,.. Điểm chung của các loại dữ liệu trong nhóm này là tính liên quan tuần tự của chúng: những quan sát phía sau thường có liên quan mật thiết đến những quan sát phía trước. \textit{Mạng nơ-ron hồi quy (recurrent neural network)}\cite{rnnorigin} gọi tắt là \textit{RNN} là một nhánh của Mạng nơ-ron nhân tạo được thiết kế đặc biệt cho việc mô hình hóa loại dữ liệu này.

%Ví dụ như trong câu sau \textit{Sư tử là loài động vật ăn ___}. Dễ biết được rằng từ trong chỗ trống sẽ là \textit{thịt}. Điều này có nghĩa là thông tin về cuối cùng trong câu được 

%Khả năng mô hình hóa dữ liệu chuỗi của RNN có thể được miêu tả như cách hoạt động của não người. Hình dung tại một thời điểm nào đó, não người hoạt động như một hàm máy tính: nó nhận \textit{input} là thông tin của các giác quan và tạo ra \textit{output} dưới dạng hành động (thể hiện ra bên ngoài) và suy nghĩ (nội tại). Lúc còn bé, khi nhìn thấy một con rắn, chúng ta sẽ nghĩ đến "rắn" - một loài vật đáng sợ. Chính suy nghĩ về loài vật này khiến chúng ta nghĩ đến việc "chạy". Rõ ràng "rắn" và "chạy" không phải là những suy nghĩ độc lập mà chúng có một thứ tự. Quan trọng hơn, chính hàm tạo ra suy nghĩ "rắn" từ hình ảnh con rắn cũng biến suy nghĩ "rắn" thành suy nghĩ "chạy". Với ví dụ này, có thể xem não người là một hàm \textit{hồi quy} - được định nghĩa là một hàm tương tự áp dụng cho mọi phần tử của một chuỗi, với đầu ra của nó phụ thuộc vào đầu vào hiện tại và các đầu vào trước đó. Cách hoạt động của mạng nơ-ron hồi quy cũng được mô phỏng dựa trên cơ chế này.

Một mạng nơ-ron hồi quy nhận đầu đầu vào là một chuỗi các vector $x_1, x_2,.., x_n$. Tại thời điểm $t (1 \le t \le n)$ vector $x_t$ thuộc chuỗi đầu vào sẽ được đưa vào mạng nơ-ron hồi quy. RNN xử lý vector đó và cập nhật trạng thái ẩn nội tại của nó được đại diện bởi vector $h_t$. Có thể hình dung $h_t$ như là một bộ nhớ lưu giữ thông tin về các vector mà nó đã xử lý cho đến thời điểm $t$. Ở dạng cơ bản nhất, công thức cập nhật trạng thái ẩn của RNN có dạng:
$$h_t = f \left(x_t, h_{t-1} \right)$$

Trong công thức trên, hàm $f$ tính trạng thái ẩn ở thời điểm $t$ dựa trên đầu vào $x_t$ và trạng thái ẩn tại thời điểm trước đó $h_{t-1}$. Thông thường, hàm $f$ là một hàm phi tuyến như hàm \textit{sigmoid} \cite{sigmoidfunction} hay hàm \textit{tanh} \cite{tanhfunction}.
$$h_t = tanh \left(W_{xh} x_t + W_{hh}h_{t-1} \right)$$

Với $W_{xh}$ và $W_{hh}$ là tham số của mô hình dưới dạng những ma trận số thực. Trạng thái ẩn bắt đầu $h_0$ có thể được khởi tạo bằng 0 hoặc là một vector chứa tri thức có sẵn như trường hợp của bộ giải mã như chúng tôi đã đề cập trong chương 1.




\section{Long short-term memory}

\section{Kiến trúc bộ mã hóa - bộ giải mã}
\subsection{Bộ mã hóa}
\subsection{Bộ giải mã}




	\chapter{Cơ chế Attention cho mô hình Dịch máy}
\ifpdf
    \graphicspath{{Chapter3/Chapter3Figs/PNG/}{Chapter3/Chapter3Figs/PDF/}{Chapter3/Chapter3Figs/}}
\else
    \graphicspath{{Chapter3/Chapter3Figs/EPS/}{Chapter3/Chapter3Figs/}}
\fi
\label{chap_3}
\begin{quote}
\textit{Chương này trình bày về cơ chế Attention. Ở đây, chúng tôi tập trung tìm hiểu về các phiên bản của cơ chế Attention và đánh giá chúng dựa trên cơ sở Toán học. Cụ thể, chúng tôi tìm hiểu về hai phiên bản Toàn cục (Global) và Cục bộ (Local):
\begin{itemize}
	\item Toàn cục: chúng tôi nhận thấy sự hạn chế hiện có của kiến trúc Bộ mã hóa-Bộ mã hóa khi thực hiện dịch những câu dài. Do vậy, chúng tôi sử dụng cơ chế Attention phiên bản Toàn cục để giải quyết vấn đề này.
	\item Cục bộ: chúng tôi quan sát thấy rằng Attention Toàn cục vẫn còn một chút vấn đề về ý tưởng và chi phí tính toán. Với sự quan sát đó, chúng tôi hiệu chỉnh Attention Toàn cục thành phiên bản Attention Cục bộ để giải quyết những hạn chế đó.
\end{itemize}}
\end{quote}
\section{Cơ chế Attention}
Ở phần trước, chúng tôi đã trình bày về kiến trúc Bộ mã hóa-Bộ giải mã cùng với những điểm mạnh của nó trong việc giải quyết bài toán Dịch máy. Tuy nhiên, kiến trúc này vẫn còn tồn tại hạn chế về việc dịch những câu dài do những thông tin được mã hóa của câu nguồn bị mất dần theo các thời điểm về sau. Lí do mà vấn đề này tồn tại thực chất là bởi vì các mô hình LSTM được sử dụng trong Bộ mã hóa và Bộ giải mã. Bản thân mô hình LSTM chưa thật sự giải quyết hoàn toàn vấn đề "sự phụ thuộc dài hạn". Để có thể vẫn tận dụng được các mô hình LSTM mà vẫn nâng cao được chất lượng dịch, chúng tôi sử dụng cơ chế Attention.

Trước khi đi vào cách hoạt động của cơ chế Attention, chúng tôi điểm qua một chút về nguồn cảm hứng và lịch sử của cơ chế này. Cơ chế Attention được lấy cảm hứng trên cơ chế đặt sự chú ý khi quan sát sự vật, hiện tượng của thị giác con người. Khi con người quan sát một sự vật, hiện tượng nào đó bằng mắt, con người chỉ có thể tập trung vào một vùng nhất định trên sự vật, hiện tượng được quan sát để ghi nhận thông tin. Sau đó, khi cần ghi nhận thêm thông tin khác, con người sẽ di chuyển vùng tập trung lên vật thể của mắt sang vị trí khác. Những vùng lân cận xung quanh vùng tập trung sẽ bị "mờ" hơn so với vùng tập trung. Cơ chế Attention đã được ứng dụng trong lĩnh vực Thị giác máy tính từ khá lâu \cite{attentionhistory2010} \cite{attentionhistory2011}. Vào những năm gần đây, cơ chế Attention được sử dụng cho các kiến trúc mạng nơ-ron hồi quy trên bài toán Dịch máy và đã đạt được những kết quả ấn tượng.

\begin{figure}
	\centering
	\includegraphics[width=0.8\textwidth]{Attention-2}
	\caption[Minh họa cơ chế Attention.]{Minh họa cơ chế Attention. Một tầng Attention được đặt ở trước bước dự đoán đầu ra của bộ giải mã.}
	\label{fig_Attention}
\end{figure}
Cơ chế Attention được sử dụng trong đề tài này là một cơ chế sử dụng thông tin trong các trạng thái ẩn của RNN trong bộ mã hóa khi thực hiện quá trình giải mã. Cụ thể là:
\begin{itemize}
	\item Trong quá trình giải mã, trước khi dự đoán đầu ra, bộ giải mã nhìn vào các thông tin nằm trong các trạng thái ẩn của RNN ở bộ mã hóa.
	\item Ở mỗi phần tử đầu ra tại thời điểm $t$, bộ giải mã dựa vào trạng thái ẩn tại thời điểm $t$ hiện tại và quyết định sử dụng các thông tin trong trạng thái ẩn ở bộ mã hóa như thế nào.
\end{itemize}
2 phiên bản Toàn cục và Cục bộ mà trong khóa luận này chúng tôi trình bày là 2 cách mà cơ chế Attention sử dụng các trạng thái ẩn của RNN trong bộ mã hóa.
Để làm rõ hơn về ý tưởng của cơ chế Attention, dưới đây chúng tôi sẽ trình bày chi tiết về nền tảng Toán học của nó.
Attention sử dụng thêm một số đại lượng:
\begin{itemize}
	\item $a_t$: trọng số gióng hàng, $a_t$ được tính theo công thức dưới đây:
	\begin{equation}
	a_t = \text{align}(h_t, \bar{h}_s) = \frac{\exp\left(\text{score}(h_t, \bar{h}_s)\right)}{\sum_{s^{'}}\exp\left(\text{score}(h_t, \bar{h}_{s^{'}})\right)}
	\end{equation}
	$a_t$ là một véc-tơ chứa các điểm số giữa trạng thái ẩn ở thời điểm $t$ $h_t$ và các trạng thái ẩn ở câu nguồn $\bar{h}_s$. Hàm điểm số score mà chúng tôi sử dụng là gồm 2 hàm:
	\begin{equation}
	\text{score}(h_t, \bar{h}_s) = \left\{
			\begin{array}{ll}
			h^T_t\bar{h}_s \ \quad\quad dot\\
			h^T_tW_a\bar{h}_s	\quad general
			\end{array}
		\right.
	\end{equation}
	// TODO: Phân tích 2 hàm
	\item $c_t$: véc-tơ ngữ cảnh tại thời điểm $t$, là trung bình có trọng số của các trạng thái ẩn ở câu nguồn:
	\begin{equation}
	c_t = \sum_{s}a_{ts}h_s
	\end{equation}
	// TODO: Phân tích ý tưởng của véc-tơ ngữ cảnh. 
	\item $\tilde{h}_t$, véc-tơ attention tại thời điểm $t$, được tính như sau:
	\begin{equation}
	\boldsymbol{\tilde{h}_t} = \tanh(\bm{W_c}[\bm{c_t};\bm{h_t}])
	\end{equation}
\end{itemize}
Bước dự đoán đầu ra không thay đổi ngoài trạng thái ẩn $\bm{h_t}$ được thay thế bởi véc-tơ attention $\bm{\tilde{h}_t}$. $\bm{\tilde{h}_t}$ được đưa qua tầng softmax để cho ra phân bố xác suất dự đoán trên các từ:
\begin{equation}
p(y_t | y_{<t}, x) = \text{softmax}(\bm{W_s\tilde{h}})
\end{equation}
Nói một cách đơn giản, mục tiêu của cơ chế Attention là xoay quanh việc tìm véc-tơ ngữ cảnh $c_t$ một cách hiệu quả.
Tiếp theo, chúng tôi trình bày chi tiết hơn về 2 phiên bản Toàn cục và Cục bộ. 2 phiên bản này chỉ khác nhau về cách suy ra véc-tơ ngữ cảnh $\bm{c_t}$, còn các bước còn lại giống nhau.
Quy trình tính toán của cơ chế Attention: $h_t -> a_t -> c_t -> \tilde{h}_t$
\section{Attention Toàn cục}
Ý tưởng của Attention toàn cục là nhìn vào toàn bộ các vị trí nguồn (các trạng thái ẩn của RNN ở bộ mã hóa) khi thực hiện giải mã.
Khi đó trọng số gióng hàng $a_t$ là một véc-tơ có kích thước thay đổi và bằng số trạng thái ẩn (số từ) ở câu nguồn: $\text{len}(a_t) = S$.

\begin{equation}
a_t = \text{align}(h_t, \bar{h}_s) = \frac{\exp\left(\text{score}(h_t, \bar{h}_s)\right)}{\sum^{S}_{s^{'}=1}\exp\left(\text{score}(h_t, \bar{h}_{s^{'}})\right)}
\end{equation}

\begin{figure}
	\centering
	\includegraphics[width=0.8\textwidth]{Global-Attention_2.png}
	\caption[Minh họa cơ chế Attention Toàn cục.]{Minh họa cơ chế Attention Toàn cục. Tại thời điểm $t$, bộ giải mã nhìn vào toàn bộ các vị trí nguồn.}
	\label{fig_Global_Attention}
\end{figure}

Ưu điểm của phương pháp này là ý tưởng đơn giản, dễ cài đặt nhưng vẫn đạt được hiệu quả tốt (sẽ được trình bày ở phần thực nghiệm). Tuy nhiên, ý tưởng này vẫn còn chưa thực sự tự nhiên và còn hạn chế. Khi dịch một từ thì không cần phải đặt "sự chú ý" lên toàn bộ câu nguồn, chỉ cần đặt "sự chú ý" lên một số từ cần thiết. Mặc dù khi mô hình Attention Toàn cục được huấn luyện tốt thì hoàn toàn có thể chỉ đặt "sự chú ý" lên một số từ thật sự cần thiết, nhưng dễ thấy rằng bản thân mô hình vẫn phải tiêu tốn chi phí cho việc tính toán trọng số gióng hàng $a_t$ cho những vị trí không cần thiết. Đó là trường hợp lý tưởng cho mô hình Attention Toàn cục, nhưng trong thực tế, để đạt được độ chính xác như thế thì phải tiêu tốn nhiều tài nguyên cho việc huấn luyện mô hình như tài nguyên về tập dữ liệu đủ lớn, đủ tốt hay thời gian huấn luyện phải đủ lâu.
Để giải quyết hạn chế trên của Attention Toàn cục, chúng tôi đã tìm hiểu và sử dụng phiên bản tinh tế hơn, đó là mô hình Attention Cục bộ. Ở phần tiếp theo, chúng tôi sẽ trình bày về mô hình này.

\section{Attention Cục bộ}
Như đã nêu ở phần trước, Attention Toàn cục có một hạn chế là đặt "sự chú ý" lên toàn bộ các từ ở câu nguồn khi dịch từng từ ở câu đích. Điều này gây tiêu tốn chi phí tính toán và có thể tạo ra những câu dịch không thực tế khi dịch những câu dài như trong các đoạn văn hay trong một tài liệu. Attention Cục bộ ra đời để giải quyết hạn chế này.

Khi dịch mỗi từ ở câu đích, Attention Cục bộ chỉ đặt "sự chú ý" lên một số từ gần nhau ở câu nguồn. Mô hình này lấy cảm hứng từ sự đánh đổi giữa 2 mô hình "soft attention" và "hard attention" được đề xuất trong công trình Show, Attend and Tell \cite{showattendandtellXu2015} để giải quyết bài toán Phát sinh câu miêu tả cho ảnh (Image Captioning). Trong công trình \cite{showattendandtellXu2015}, Attention Toàn cục tương ứng với "soft attention", "sự chú ý" được đặt trên toàn bộ bức ảnh. Còn "hard attention" thì đặt "sự chú ý" lên một số phần của bức ảnh.

Dễ thấy, với cách hoạt động chỉ tập trung một số các từ gần nhau ở câu nguồn, mô hình hoạt động gần với cách con người tập trung vào một sự vật, hiện tượng nào đó. Chi phí cho huấn luyện và dự đoán sẽ được giảm bớt bởi vì chúng ta chỉ thực hiện tính véc-tơ trọng số gióng hàng $a_t$ cho những từ mà mô hình đặt "sự chú ý" lên.

Để làm rõ hơn về cách thức hoạt động của mô hình Attention Cục bộ, chúng tôi sẽ trình bày cụ thể hơn về nền tảng Toán học của mô hình này. Bên cạnh những đại lượng đã có ở mô hình Attention Toàn cục, Attention Cục bộ có thêm và thay đổi một số đại lượng như sau:
\begin{itemize}
	\item $p_t$: vị trí đã được gióng hàng. Tại mỗi thời điểm $t$, mô hình sẽ phát sinh một số thực $p_t$. Số thực này có giá trị nằm trong đoạn $[0, S]$ với ý nghĩa rằng đây là vị trí đã được gióng hàng của với từ ở câu nguồn tại thời điểm $t$ hiện tại. Hay nói cách khác, "sự chú ý" được đặt trên từ có vị trí $p_t$ này. Để ý thấy rằng có sự không tự nhiên khi $p_t$ là một số thực, do vậy $p_t$ không thể cho biết được chính xác từ nào sẽ được đặt "sự chú ý" lên. Thực tế, với miền giá trị số thực, $p_t$ có tác dụng là dùng để làm vị trí trung tâm cho các từ lân cận. Để làm rõ hơn về vấn đề này, chúng tôi sẽ trình bày rõ ràng hơn ở sau.
	\item Đối quá trình tính véc-tơ ngữ cảnh $c_t$ có sự thay đổi rằng mô hình xét các vị trí ở câu nguồn mà nằm xung quanh vị trí $p_t$ một đoạn $D$. $D$ là một đại lượng với miền số nguyên lớn hơn 0 và được gọi là kích thước cửa sổ. Cụ thể:
	\begin{equation}
	c_t = \sum_{x \in [p_t - D, p_t + D]} a_{tx}\tilde{h}_x
	\end{equation}
	$D$ là một siêu tham số của mô hình. Việc lựa chọn giá trị của $D$ là dựa vào thực nghiệm. Theo đề xuất của \cite{attentionThangLuong2015}, chúng tôi lựa chọn $D = 10$.
\end{itemize}

Mô hình Attention Cục bộ có 2 biến thể:
\begin{itemize}
	\item Gióng hàng đều (monotonic alignment - local-m): vị trí được gióng hàng được phát sinh một cách đơn giản bằng cách cho $p_t = t$ tại mỗi thời điểm $t$. Ta giả định rằng các từ ở câu nguồn và các từ ở câu đích được gióng hàng đều nhau theo từng từ.
	\item Gióng hàng dự đoán (predictive alignment - local-p): giả định rằng tất cả từ ở câu nguồn và câu đích đều được gióng hàng đều nhau không thực tế vì giữa 2 ngôn ngữ có ngữ pháp riêng và trật tự từ khác nhau. Chúng tôi sẽ trình bày rõ ràng hơn vào các phần sau. Do vậy, mô hình sẽ phát sinh vị trí được gióng hàng $p_t$ một cách tự nhiên hơn cho phù hợp đặc điểm của ngôn ngữ. Cụ thể mô hình sẽ phát sinh vị trí $p_t$ tại mỗi thời điểm $t$ như sau:
	\begin{equation}
	p_t = S \cdot \text{sigmoid} (v^T_p \tanh(W_ph_t))
	\end{equation}
\end{itemize}

\subsection{Thuật toán ``Sleep-Wake Stochastic Gradient Descent''}
Để khắc phục khó khăn của việc huấn luyện SRAEs với chuẩn L1, chúng tôi đề xuất một phiên bản điều chỉnh của thuật toán ``Stochastic Gradient Descent'' (SGD), gọi là ``Sleep-Wake Stochastic Gradient Descent'' (SW-SGD). Ý tưởng là trong mỗi ``epoch'' của SGD (một ``epoch'' ứng một lần duyệt qua tất cả các mẫu trong tập huấn luyện), ta tính tổng giá trị đầu ra của mỗi nơ-ron ẩn; và sau mỗi ``epoch'', ta kiểm xem có nơ-ron nào ``ngủ'' không (có tổng giá trị đầu ra bằng không) và ``đánh thức'' chúng bằng cách khởi tạo lại véc-tơ trọng số đi vào. Mặc dù cách làm này rất đơn giản, nhưng thí nghiệm của chúng tôi cho thấy nó có thể giúp SRAEs học được thành công các đặc trưng mà không có đặc trưng nào ``ngủ''.

Một cách cụ thể, cho tập huấn luyện không có nhãn $\{x^{(1)}, x^{(2)}, \ldots, x^{(N)}\}$, thuật toán SW-SGD dùng để cực tiểu hóa hàm chi phí sau của SRAEs:
\begin{equation}
\begin{split}
	C(W) &= \frac{1}{N}\sum_{i=1}^N C^{(i)}(W)\\
		 &= \frac{1}{N}\sum_{i=1}^N\left(\|x^{(i)} - \hat{x}^{(i)}\|_2^2 + \lambda \|h^{(i)}\|_1\right)
\end{split}
\end{equation}
Trong đó:
\begin{itemize}
	\item $h^{(i)}$ là véc-tơ đầu ra ở tầng ẩn tương ứng với véc-tơ đầu vào $x^{(i)}$: \[h^{(i)} = \max\left(0, W^{(e)}x^{(i)} + b^{(e)}\right)\]
	\item $\hat{x}^{(i)}$ là véc-tơ tái tạo của véc-tơ đầu vào $x^{(i)}$: \[\hat{x}^{(i)} = W^{(d)}h^{(i)} + b^{(d)}\]
	\item $W=\{W^{(e)}, b^{(e)}, W^{(d)}, b^{(d)}\}$ là các tham số của SRAEs.
\end{itemize}
Từng bước của thuật toán SW-SGD được trình bày ở thuật toán \ref{alg_SW-SGD} (những chỗ thay đổi so với thuật toán SGD ban đầu được \emph{in nghiêng}).
%\begin{leftbar}
%\begin{enumerate}
%	\item Khởi tạo ngẫu nhiên cho $W$.
%	\item Lặp cho đến khi thỏa điều kiện dừng:
%	\begin{itemize}
%		\item \emph{Khởi tạo véc-tơ $s$ gồm có $D_h$ phần tử (với $D_h$ là số lượng nơ-ron ẩn của SRAEs), trong đó mỗi phần tử có giá trị bằng $0$ (phần tử $s_j$ của véc-tơ $s$ dùng để lưu tổng giá trị đầu ra của nơ-ron ẩn $j$ với tất cả các mẫu trong tập huấn luyện sau một ``epoch'').}
%		\item Xáo trộn ngẫu nhiên thứ tự của các mẫu trong tập huấn luyện (thường sẽ giúp hội tụ nhanh hơn).
%		\item Với ``mini-batch'' thứ $b=1,2,\ldots,\frac{N}{B}$ ($N$ là số lượng mẫu trong tập huấn luyện, $B$ là kích thước của ``mini-batch''):
%		\begin{itemize}
%			\item Với mẫu huấn luyện thứ $i=(b-1)B+1, (b-1)B+2, \ldots, bB$:
%			\begin{itemize}
%				\item Lan truyền tiến với véc-tơ đầu vào $x^{(i)}$.
%				\item \emph{Cập nhật véc-tơ $s$: $s = s + h^{(i)}$ (với $h^{(i)}$ là véc-tơ đầu ra ở tầng ẩn tương ứng với véc-tơ đầu vào $x^{(i)}$).}
%				\item Lan truyền ngược và tính véc-tơ đạo hàm riêng $\nabla C^{(i)}(W)$.
%			\end{itemize}
%			\item Cập nhật $W$:
%			\[W = W - \alpha \frac{1}{B} \sum_{i=(b-1)B+1}^{bB} \nabla C^{(i)}(W)\]
%			($\alpha > 0$ là hệ số học)
%		\end{itemize}
%		\item \emph{Kiểm xem có nơ-ron ẩn nào ``ngủ'' (có tổng đầu ra $s_j=0$) và ``đánh thức'' bằng cách khởi tạo lại véc-tơ trọng số đi vào nơ-ron ẩn này.}
%	\end{itemize}
%\end{enumerate}
%\end{leftbar}
\begin{algorithm}
	\newalgname{Thuật toán}
	\caption{Sleep-Wake Stochastic Gradient Descent (SW-SGD)}
	\label{alg_SW-SGD}
	\begin{algorithmic}[1]
		\renewcommand{\algorithmicrequire}{\textbf{Đầu vào:}}
		\renewcommand{\algorithmicensure}{\textbf{Đầu ra:}}
		\algnewcommand\algorithmicoperation{\textbf{Thao tác:}}
		\algnewcommand\Operation{\item[\algorithmicoperation]}
		
		\Require Tập huấn luyện không có nhãn $\{x^{(1)},\ldots,x^{(N)}\}$, hệ số học $\alpha>0$, kích thước ``mini-batch'' B
		\Ensure Bộ tham số $W$ của SRAEs để cho hàm chi phí $C(W)$ đạt cực tiểu
		
		\Operation
		\State Khởi tạo ngẫu nhiên cho $W$
		\While{chưa thỏa điều kiện dừng} \%\% Với mỗi ``epoch''
			\State \parbox[t]{\dimexpr\linewidth-\algorithmicindent}{\emph{Khởi tạo véc-tơ $s$ gồm có $D_h$ phần tử (với $D_h$ là số lượng nơ-ron ẩn của SRAEs), trong đó mỗi phần tử có giá trị bằng $0$ (phần tử $s_j$ của véc-tơ $s$ dùng để lưu tổng giá trị đầu ra của nơ-ron ẩn $j$ với tất cả các mẫu trong tập huấn luyện sau một ``epoch'')}\strut}
			\State \parbox[t]{\dimexpr\linewidth-\algorithmicindent}{Xáo trộn ngẫu nhiên thứ tự của các mẫu trong tập huấn luyện (thường sẽ giúp hội tụ nhanh hơn)\strut}
			\For{b = 1 : N/B} \%\% Với mỗi ``mini-batch''
				\For{i = (b-1)B+1 : bB} \%\% Với mỗi mẫu huấn luyện trong ``mini-batch''
					\State Lan truyền tiến với véc-tơ đầu vào $x^{(i)}$
					\State \parbox[t]{\dimexpr\linewidth-\algorithmicindent}{\emph{Cập nhật véc-tơ $s$: $s = s + h^{(i)}$ \\(với $h^{(i)}$ là véc-tơ đầu ra ở tầng ẩn tương ứng với véc-tơ đầu vào $x^{(i)}$)}\strut}
					\State Lan truyền ngược và tính véc-tơ đạo hàm riêng $\nabla C^{(i)}(W)$
				\EndFor
				\State Cập nhật $W$: $W = W - \alpha \frac{1}{B} \sum_{i=(b-1)B+1}^{bB} \nabla C^{(i)}(W)$ 
			\EndFor
			\State \parbox[t]{\dimexpr\linewidth-\algorithmicindent}{\emph{Kiểm xem có nơ-ron ẩn nào ``ngủ'' (có tổng đầu ra $s_j=0$) và ``đánh thức'' bằng cách khởi tạo lại véc-tơ trọng số đi vào nơ-ron ẩn này}}
		\EndWhile
	\end{algorithmic}
\end{algorithm}

\section{Ràng buộc trọng số trong SRAEs}
Bên cạnh ràng buộc thưa, ràng buộc trọng số cũng là một thành phần quan trọng để làm cho SAEs ``hoạt động''. Tại sao cần phải ràng buộc trọng số? Ví dụ, trong Sparse Coding, ta cần phải ràng buộc các véc-tơ cơ sở được chuẩn hóa (có chiều dài bằng 1); nếu không thì sẽ xảy ra trường hợp là giá trị của hàm chi phí ở công thức (\ref{eq_SparseCoding}) có thể được làm giảm xuống một cách ``tầm thường'' bằng cách chia hệ số cho một số lớn tùy ý và nhân véc-tơ cơ sở tương ứng với cùng số lớn đó (làm như vậy sẽ làm độ thưa giảm xuống tùy ý, còn độ lỗi tái tạo thì giữ nguyên). Các véc-tơ cơ cở trong ``Sparse Coding'' tương ứng với các cột của ma trận trọng số $W^{(d)}$ của bộ giải mã của SAEs (mỗi cột của $W^{(d)}$ ứng với véc-tơ trọng số đi ra tại mỗi nơ-ron ẩn). Như vậy, trong SAEs, ta cũng có thể ràng buộc mỗi cột của $W^{(d)}$ được chuẩn hóa (có chiều dài bằng 1) giống như ở Sparse Coding. Nhưng còn ma trận trọng số $W^{(e)}$ của bộ mã hóa của SAEs? Ta nên ràng buộc $W^{(e)}$ như thế nào cho hợp lý?

Dưới đây là một số cách đã được đề xuất để ràng buộc trọng số của SAEs:
\begin{itemize}
	\item \textbf{Ràng buộc $W^{(d)} = (W^{(e)})^T$}: bộ trọng số được dùng chung cho cả bộ mã hóa và bộ giải mã (cụ thể là $W^{(d)}$ và $W^{(e)}$ là chuyển vị của nhau) \cite{coates2012demystifying}. Cách ràng buộc trọng số này cũng được dùng trong các loại ``Auto-Encoders'' khác như ``Denoising Auto-Encoders'' và ``Contractive Auto-Encoders'' \cite{vincent2008extracting}\cite{rifai2011contractive}\cite{rifai2011HCAEs}. Lưu ý là tất cả \cite{coates2012demystifying}\cite{vincent2008extracting}\cite{rifai2011contractive}\cite{rifai2011HCAEs} đều dùng hàm kích hoạt sigmoid ở tầng ẩn. Trong trường hợp dùng hàm kích hoạt tuyến tính ($f(x) = x$) ở tầng ẩn, ràng buộc $W^{(d)} = (W^{(e)})^T$ sẽ có xu hướng làm cho các véc-tơ cơ sở (các dòng của $W^{(e)}$ hay các cột của $W^{(d)}$) trực giao với nhau và được chuẩn hóa \cite{le2011ica}; nhưng trong trường hợp dùng hàm kích hoạt sigmoid ở tầng ẩn, ta không rõ chuyện gì đang xảy ra. Một điểm lợi của việc dùng chung bộ trọng số là tiết kiệm bộ nhớ lưu trữ; điều này sẽ có ích khi cài đặt song song trên GPU (Graphical Processing Units).
	\item \textbf{Ràng buộc $W^{(d)}$ được chuẩn hóa}: các cột của $W^{(d)}$ được ràng buộc là có độ dài bằng 1 \cite{zeiler2013rectified}. Ràng buộc này tương tự như ở ``Sparse Coding'' và giúp ngăn chặn việc hàm chi phí có thể bị làm giảm xuống một cách ``tầm thường'' như đã nói ở trên. Nhưng còn bộ trọng số $W^{(e)}$ của bộ mã hóa? Chẳng hạn, để công bằng giữa các đặc trưng, ta cũng nên ràng buộc các véc-tơ dòng của $W^{(e)}$ (ứng với các véc-tơ trọng số đi vào các nơ-ron ẩn; các véc-tơ này đóng vai trò như các bộ lọc đặc trưng) có cùng độ dài.
	\item \textbf{Ràng buộc các trọng số có giá trị bình phương nhỏ (weight decay)}: các trọng số của cả bộ mã hóa và bộ giải mã đều được ràng buộc là có độ lớn nhỏ bằng cách phạt tổng bình phương của chúng \cite{goodfellow2009measuring}\cite{coates2011analysis}. Cách ràng buộc này vốn ban đầu được dùng trong mạng nơ-ron học có giám sát để tránh vấn đề quá khớp. Khi áp dụng cho SAEs, ta có hiểu nó là một phiên bản ``mềm'' của cách ràng buộc $W^{(d)}$ được chuẩn hóa ở trên và nhờ đó cũng sẽ giúp cho SAEs tránh khỏi tình trạng hàm chi phí bị giảm xuống một cách ``tầm thường''; ngoài ra, nó còn ràng buộc thêm là các véc-tơ dòng của $W^{(e)}$ (ứng với các bộ lọc đặc trưng) có độ dài xấp xỉ bằng nhau (đều nhỏ). Tuy nhiên, cách ràng buộc này lại làm xuất hiện thêm một siêu tham số; ta không muốn điều này.
\end{itemize}

\subsection{Cách ràng buộc trọng số đề xuất cho SRAEs}
Với SRAEs (SAEs sử dụng hàm kích hoạt ``rectified linear'' ở tầng ẩn), không rõ là ta nên sử dụng cách ràng buộc trọng số nào trong những cách ở trên. Trong phần này, chúng tôi đề xuất một cách ràng buộc trọng số mới và hợp lý cho SRAEs. Cách ràng buộc này không đưa thêm siêu tham số nào. Cụ thể là, cách ràng buộc trọng số của chúng tôi bao gồm đồng thời hai ràng buộc:
\begin{itemize}
	\item Thứ nhất, chúng tôi ràng buộc ma trận trọng số $W^{(e)}$ của bộ mã hóa và ma trận trọng số $W^{(d)}$ của bộ giải mã là chuyển vị của nhau: $W^{(d)} = (W^{(e)})^T$.
	\item Thứ hai, chúng tôi đồng thời cũng ràng buộc là các véc-tơ dòng của $W^{(e)}$ và các véc-tơ cột của $W^{(d)}$ được chuẩn hóa (có độ dài bằng 1). Ở đây, mỗi véc-tơ dòng của $W^{(e)}$ ứng với véc-tơ trọng số đi vào ở mỗi nơ-ron ẩn, và mỗi véc-tơ cột của $W^{(d)}$ ứng với véc-tơ trọng số đi ra ở mỗi nơ-ron ẩn.
\end{itemize}

Với một véc-tơ đầu vào $x$, nếu ta chỉ chú ý đến các nơ-ron được ``bật'' (có giá trị đầu ra khác không) ở tầng ẩn thì đây là một hệ thống tuyến tính (minh họa ở hình \ref{fig_SRAE}). Do đó, với hai ràng buộc ở trên, SRAEs sẽ chiếu véc-tơ đầu vào $x$ xuống một hệ trục tọa độ cục bộ sao cho từ hệ trục tọa độ này có thể tái tạo được tốt véc-tơ $x$ ban đầu; hệ trục tọa độ cục bộ này bao gồm một số ít các véc-tơ cơ sở (đã được chuẩn hóa) được chọn lựa bởi hàm ``rectified linear'' từ tập lớn các véc-tơ cơ sở. Ta có thể hiểu tập con các véc-tơ cơ sở này biểu diễn vùng không gian PCA cục bộ xung quanh $x$. Như vậy, SRAEs (với ràng buộc trọng số đề xuất của chúng tôi) có thể học được mặt phi tuyến mà ở đó dữ liệu tập trung bằng cách ghép nhiều mặt tuyến tính cục bộ lại với nhau (minh họa ở hình \ref{fig_local_charts}). Mỗi mặt tuyến tính cục bộ được phụ trách bởi một tập con các véc-tơ cơ sở. Điểm lợi ở đây là các véc-tơ cơ sở có thể được dùng chung giữa các mặt tuyến tính cục bộ láng giềng nhau.

\begin{figure}
	\centering
	\includegraphics[width=0.6\textwidth]{SRAE}
	\caption[Minh họa SRAEs]{Minh họa SRAEs. Với một véc-tơ đầu vào, nếu ta chỉ chú ý đến các nơ-ron được ``bật'' (có giá trị đầu ra khác không) ở tầng ẩn thì đây là một hệ thống tuyến tính.}
	\label{fig_SRAE}
\end{figure}
\begin{figure}
	\centering
	\includegraphics[width=\textwidth]{local_charts}
	\caption[Minh họa mặt phi tuyến mà SRAEs học được]{Với một véc-tơ đầu vào $x$, chỉ có một tập con các nơ-ron ẩn được bật. Tập con các véc-tơ cơ sở tương ứng với tập con các nơ-ron ẩn này biễu diễn một vùng không gian cục bộ xung quanh $x$ (giống như vùng không gian PCA cục bộ). Như vậy, SRAEs (với ràng buộc trọng số đề xuất của chúng tôi) có thể học được mặt phi tuyến mà ở đó dữ liệu tập trung bằng cách ghép nhiều mặt tuyến tính cục bộ lại với nhau. Mỗi mặt tuyến tính cục bộ được phụ trách bởi một tập con các véc-tơ cơ sở. Điểm lợi ở đây là các véc-tơ cơ sở có thể được dùng chung giữa các mặt tuyến tính cục bộ láng giềng nhau.}
	\label{fig_local_charts}
\end{figure}

Như vậy, ta cần phải tối thiểu hóa hàm chi phí của SRAEs với hai ràng buộc trọng số ở trên. Trong khi ràng buộc thứ nhất ($W^{(d)} = (W^{(e)})^T$) có thể được tích hợp dễ dàng vào thuật toán tối ưu hóa ``Stochastic Gradient Descent''  (SGD), ràng buộc thứ hai (các dòng của $W^{(e)}$ và các cột của $W^{(d)}$ được chuẩn hóa) thoạt nhìn khó có thể tích hợp vào thuật toán SGD và có thể ta cần phải sử dụng đến các phương pháp tối ưu hóa phức tạp hơn. Để giải quyết vấn đề này, chúng tôi thay đổi công thức lan truyền tiến của SRAEs như sau:
\begin{equation}
	h = \max(0, \hat{W}^{(e)}x + b^{(e)})
\end{equation}
\begin{equation}
	\hat{x} = (\hat{W}^{(e)})^Th + b^{(d)}
\end{equation}
Trong đó, ma trận $\hat{W}^{(e)}$ là ma trận $W^{(e)}$ với các dòng đã được chuẩn hóa (bằng cách lấy mỗi phần tử trên một dòng của $W^{(e)}$ chia cho căn bậc hai của tổng bình phương của tất cả các phần tử trên dòng đó). Ở đây, các tham số được học vẫn là $W^{(e)}$, $b^{(e)}$, và $b^{(d)}$. Bằng cách này, ta vẫn có thể sử dụng thuật toán SGD như bình thường. Khi đưa thêm bước chuẩn hóa vào công thức lan truyền tiến như vậy, ta cũng cần phải tính lại các đạo hàm riêng của hàm chi phí theo các tham số (sẽ phức tạp hơn so với công thức lan truyền tiến ban đầu). Chúng tôi sử dụng ngôn ngữ lập trình là Theano \cite{bergstra+al:2010-scipy}; nhờ tính năng tính đạo hàm một cách tự động của Theano, ở đây ta sẽ không cần phải tính toán cụ thể công thức của các đạo hàm riêng này.

	\chapter{Các Kết Quả Thí Nghiệm}
\ifpdf
    \graphicspath{{Chapter4/Chapter4Figs/PNG/}{Chapter4/Chapter4Figs/PDF/}{Chapter4/Chapter4Figs/}}
\else
    \graphicspath{{Chapter4/Chapter4Figs/EPS/}{Chapter4/Chapter4Figs/}}
\fi
\label{chap_4}
\begin{quote}
\textit{Trong chương này, chúng tôi trình bày các kết quả thí nghiệm để đánh giá các đề xuất đã được nói ở chương trước. Bộ dữ liệu được dùng để tiến hành các thí nghiệm là bộ MNIST (bộ ảnh chữ số viết tay gồm các chữ số từ 0 đến 9). Các kết quả thí nghiệm cho thấy khi huấn luyện ``Sparse Rectified Auto-Encoders'' (SRAEs) với chuẩn L1 sẽ gặp phải vấn đề nơ-ron ``ngủ'', và chiến lược ``ngủ - đánh thức'' trong thuật toán ``Sleep-Wake Stochastic Gradient Descent'' (SW-SGD) của chúng tôi có thể giúp khắc phục vấn đề này. Các kết quả thí nghiệm cũng cho thấy cách ràng buộc trọng số đề xuất của chúng tôi cho kết quả tốt nhất trong số các cách ràng buộc trọng số có thể áp dụng cho SRAEs. Cuối cùng, thí nghiệm cũng cho thấy SRAEs với hai đề xuất trên của chúng tôi (SW-SGD và cách ràng buộc trọng số) có thể học được những đặc trưng cho kết quả phân lớp tốt khi so sánh với các loại ``Auto-Encoders'' khác.}
\end{quote}
\section{Các thiết lập thí nghiệm}
Chúng tôi tiến hành các thí nghiệm trên bộ dữ liệu MNIST \cite{MNIST}; bộ dữ liệu này gồm các ảnh xám (có kích thước $28 \times 28$) của mười chữ số viết tay từ 0 đến 9. Ở hình \ref{fig_MNIST} là một số ảnh mẫu của bộ dữ liệu này. Dữ liệu được tiến hành tiền xử lý bằng cách lấy mỗi giá trị điểm ảnh chia cho $255$ để đưa về đoạn $[0, 1]$. Chúng tôi sử dụng cách phân chia thường được sử dụng cho bộ dữ liệu này: $50000$ ảnh dùng để huấn luyện, $10000$ ảnh dùng để chọn các siêu tham số (validation), và $10000$ ảnh dùng để kiểm tra (test).

\begin{figure}
	\centering
	\includegraphics{MNIST}
	\caption{Một số ảnh mẫu của bộ dữ liệu MNIST}
	\label{fig_MNIST}
\end{figure}

Chúng tôi sử dụng ngôn ngữ lập trình Theano \cite{bergstra+al:2010-scipy} bởi vì ngôn ngữ này cho phép dễ dàng cài đặt các thuật toán và dễ dàng sử dụng GPU (Graphical Processing Units) để tính toán song song. Loại GPU mà chúng tôi sử dụng là NVIDIA GTX 560.

Sau khi tiến hành xong bước học đặc trưng không giám sát, chúng tôi đánh giá các đặc trưng học được bằng cách sử dụng chúng để huấn luyện mô hình phân lớp ``Softmax Regression'' và đo độ lỗi phân lớp. Một cách cụ thể, cho tập huấn luyện $\{(x^{(1)}, y^{(1)}), \ldots, (x^{(N)}, y^{(N)})\}$ với $x^{(i)} \in \mathbb{R}^{D_x}$ là véc-tơ điểm ảnh và $y^{(i)} \in \{0, \ldots, 9\}$ là nhãn lớp. Sau khi ``Auto-Encoder'' đã được huấn luyện trên tập không có nhãn $\{x^{(1)}, \ldots, x^{(N)}\}$, ta lần lượt đưa từng véc-tơ $x^{(i)}$ vào ``Auto-Encoder'' và thu được ở tầng ẩn véc-tơ đặc trưng tương ứng $h^{(i)}$; bằng cách này, ta có được tập huấn luyện mới $\{(h^{(1)}, y^{(1)}), \ldots, (h^{(N)}, y^{(N)})\}$. Kế đến, tập huấn luyện mới này được sử dụng để huấn luyện ``Softmax Regression''. Để dự đoán nhãn lớp cho một véc-tơ đầu vào mới $x_{test}$, đầu tiên ta sử dụng ``Auto-Encoder'' đã được huấn luyện để tính véc-tơ đặc trưng tương ứng $h_{test}$; sau đó đưa $h_{test}$ này vào ``Softmax Regression'' đã được huấn luyện để tính giá trị nhãn lớp dự đoán.

Trong cả hai giai đoạn học không giám sát và có giám sát, chúng tôi sử dụng thuật toán để để cực tiểu hóa hàm chi phí là Stochastic Gradient Descent (SGD) với kích thước của ``mini-batch'' là $100$ mẫu huấn luyện. Chiến lược ``dừng sớm'' (early stopping) được sử dụng để quyết định số vòng lặp (epoch) của SGD cũng như là để chống vấn đề quá khớp (trong giai đoạn học không giám sát, chúng tôi dừng quá trình tối ưu hóa dựa vào giá trị của hàm chi phí trên tập ``validation''; còn trong giai đoạn học có giám sát, chúng tôi dựa vào độ lỗi phân lớp trên tập ``validation''). Trong tất cả các thí nghiệm dưới đây, chúng tôi dùng SRAEs với $1000$ nơ-ron ẩn, tham số ``thỏa hiệp'' giữa độ lỗi tái tạo và độ thưa $\lambda$ bằng $0.25$, hệ số học khi học không giám sát bằng $0.05$, và hệ số học khi học có giám sát bằng $1$ (số lượng nơ-ron ẩn được chọn theo \cite{rifai2011HCAEs}, các siêu tham số còn lại được chọn dựa vào thực nghiệm).
\section{SGD và SW-SGD}
Để thấy được vấn đề gặp phải khi huấn luyện SRAEs với ràng buộc thưa bằng chuẩn L1 cũng như là tác dụng của chiến lược ``ngủ - đánh thức'' của chúng tôi, trong phần này chúng tôi so sánh việc huấn luyện SRAEs bằng thuật toán ``Stochastic Gradient Descent'' (SGD) và phiên bản điều chỉnh của chúng tôi, ``Sleep-Wake Stochastic Gradient Descent'' (SW-SGD). Trong thí nghiệm này, cách ràng buộc trọng số đề xuất của chúng tôi được sử dụng ($W^{(d)} = (W^{(e)})^T$, và các dòng của $W^{(e)}$ và các cột của $W^{(d)}$ được chuẩn hóa).

Hình \ref{fig_SGDvsSWSGD} thể hiện số lượng nơ-ron ``ngủ'' của SRAEs trong khi thực hiện quá trình tối ưu hóa hàm chi phí với SGD và SW-SGD. Vấn đề gặp phải khi huấn luyện SRAEs với chuẩn L1 là trong quá trình tối ưu hóa, chuẩn L1 có thể đẩy các véc-tơ trọng số đi vào các nơ-ron ẩn vào trạng thái ``ngủ'' (nghĩa là, nơ-ron ẩn tương ứng luôn cho giá trị đầu ra bằng 0 với tất cả các mẫu huấn luyện) và sau đó, chúng sẽ không bao giờ còn được cập nhật nữa. Như có thể thấy từ hình \ref{fig_SGDvsSWSGD}, khi sử dụng SGD, số lượng nơ-ron ``ngủ'' tăng dần trong quá trình tối ưu hóa, đặc biệt là trong những vòng lặp đầu tiên, khi mà quá trình tối ưu hóa vẫn còn chưa ổn định. Vấn đề nơ-ron ``ngủ'' này của chuẩn L1 có thể được khắc phục một cách đơn giản bằng chiến lược ``ngủ - đánh thức'' của chúng tôi; quá trình tối ưu hóa của SW-SGD kết thúc mà không có nơ-ron ``ngủ'' nào cả.
\begin{figure}
	\centering
	\includegraphics[width=0.8\textwidth]{SGD_vs_SW-SGD}
	\caption[So sánh giữa SGD với SW-SGD]{Số lượng nơ-ron ``ngủ'' của SRAEs trong khi thực hiện quá trình tối ưu hóa với SGD và với SW-SGD. Quá trình tối ưu hóa của SGD kết thúc với 228 nơ-ron ``ngủ'' trong tổng số 1000 nơ-ron; trong khi đó, SW-SGD kết thúc mà không có nơ-ron nào ``ngủ''. (Hai quá trình tối ưu hóa của SGD và SW-SGD kết thúc sau các số lượng vòng lặp khác nhau là do chiến lược ``dừng sớm''.)}
	\label{fig_SGDvsSWSGD}
\end{figure}

Ở hình \ref{fig_filters} là một số bộ lọc (một bộ lọc tương ứng với véc-tơ trọng số đi vào một nơ-ron ẩn) học được bởi SGD và SW-SGD. Như ta có thể thấy, với SGD, có 5 nơ-ron ``ngủ''; các bộ lọc của chúng nhìn vô nghĩa. Với SW-SGD, không có nơ-ron nào ``ngủ''; tất cả các bộ lọc đều nhìn có nghĩa, mỗi bộ lọc dò tìm một đường nét nào đó của chữ số.
\begin{figure}
	\centering
	\includegraphics[width=\textwidth]{some_filters}
	\caption[So sánh giữa các bộ lọc học được bởi SGD với SW-SGD]{Ở hình (a) là một số bộ lọc (một bộ lọc tương ứng với véc-tơ trọng số đi vào một nơ-ron ẩn) học được bởi SGD; ta có thể thấy có 5 bộ lọc nhìn vô nghĩa tương ứng với 5 nơ-ron ``ngủ''. Còn ở hình (b) là các bộ lọc học được bởi SW-SGD; tất cả các bộ lọc đều nhìn có nghĩa, mỗi bộ lọc dò tìm một đường nét nào đó của chữ số.}
	\label{fig_filters}
\end{figure}

Nhờ sử dụng hết tất cả các nơ-ron ẩn, SW-SGD tìm được giá trị cực tiểu của hàm chi phí của SRAEs trên tập huấn luyện tốt hơn so với SGD; và các đặc trưng học được của SW-SGD cũng cho kết quả phân lớp (với ``Softmax Regression'') trên tập kiểm tra tốt hơn so với SGD (bảng \ref{table_SGDvsSW-SGD}).

\begin{table}
	\centering
	\caption[So sánh giữa SGD với SW-SGD]{Giá trị hàm chi phí của SRAEs trên tập huấn luyện và độ lỗi phân lớp (với ``Softmax Regression'') trên tập kiểm tra khi huấn luyện SRAEs với SGD và với SW-SGD.}
	\label{table_SGDvsSW-SGD}
	\begin{tabular}{|c|c|c|} \hline
	 & SGD & SW-SGD\\ \hline
	Giá trị hàm chi phí của SRAEs trên tập huấn luyện & 9.84 & \textbf{9.48}\\ \hline 
	Độ lỗi phân lớp trên tập kiểm tra (\%) & 1.70 & \textbf{1.62}\\ \hline
	\end{tabular}
\end{table}

\section{Cách ràng buộc trọng số đề xuất của chúng tôi và các cách ràng buộc trọng số khác}
Trong thí nghiệm thứ hai này, cách ràng buộc trọng số đề xuất cho SRAEs của chúng tôi được so sánh với các cách ràng buộc trọng số khác mà có thể áp dụng cho SRAEs. Cụ thể ở đây, chúng tôi so sánh với các cách ràng buộc trọng số sau:
\begin{itemize}
	\item \textbf{$W^{(d)}$ được chuẩn hóa:} các véc-tơ cột của $W^{(d)}$ được ràng buộc là chuẩn hóa (có độ dài bằng 1); mỗi véc-tơ cột của $W^{(d)}$ tương ứng với véc-tơ trọng số đi ra ở mỗi nơ-ron ẩn.
	\item \textbf{$W^{(e)}$ và $W^{(d)}$ được chuẩn hóa:} các véc-tơ dòng của $W^{(e)}$ và các véc-tơ cột của $W^{(d)}$ được ràng buộc là chuẩn hóa (có độ dài bằng 1); mỗi véc-tơ dòng của $W^{(e)}$ và mỗi véc-tơ cột của $W^{(d)}$ lần lượt tương ứng với véc-tơ trọng số đi vào và véc-tơ trọng số đi ra ở mỗi nơ-ron ẩn.
	\item \textbf{$W^{(d)} = (W^{(e)})^T$:} $W^{(e)}$ và $W^{(d)}$ được ràng buộc là chuyển vị của nhau.
\end{itemize}
Cách ràng buộc trọng số của chúng tôi là kết hợp của hai ràng buộc: $W^{(e)}$ và $W^{(d)}$ được chuẩn hóa, và $W^{(d)} = (W^{(e)})^T$. Trong thí nghiệm này, chúng tôi dùng SW-SGD để huấn luyện SRAEs.

Như có thể thấy ở bảng \ref{table_OurWConstraintVSOtherWConstraints}, trong số các cách ràng buộc trọng số, cách ràng buộc của chúng tôi giúp SRAEs học được những đặc trưng cho kết quả phân lớp (với ``Softmax Regression'') tốt nhất trên tập kiểm tra. Ngoài ra, bảng \ref{table_OurWConstraintVSOtherWConstraints} cũng so sánh thời gian huấn luyện SRAEs trên một vòng lặp (ứng với một lần duyệt qua toàn bộ các mẫu huấn luyện) với các cách ràng buộc trọng số khác nhau này (do chiến lược ``dừng sớm'', quá trình huấn luyện SRAEs với các cách ràng buộc khác nhau có thể kết thúc sau các số lượng vòng lặp khác nhau; do đó, để chính xác, ta nên so sánh theo thời gian huấn luyện xét trên một vòng lặp hơn là tổng thời gian huấn luyện). Các cách ràng buộc trọng số được sắp xếp theo thứ tự thời gian huấn luyện (trên một vòng lặp) tăng dần là: $W^{(d)} = (W^{(e)})^T$ (2 giây), $W^{(d)}$ được chuẩn hóa (3 giây), cách ràng buộc trọng số của chúng tôi (4 giây), $W^{(e)}$ và $W^{(d)}$ được chuẩn hóa (5 giây). Thứ tự này là hợp lý:
\begin{itemize}
	\item Ràng buộc $W^{(d)} = (W^{(e)})^T$ có thời gian huấn luyện nhanh nhất vì SRAEs không phải thực hiện bước chuẩn hóa.
	\item Ràng buộc $W^{(d)}$ được chuẩn hóa có thời gian huấn luyện lâu hơn vì bộ giải mã của SRAEs phải thực hiện bước chuẩn hóa khi lan truyền tiến; và do đó, khi lan truyền ngược, việc tính toán các đạo hàm riêng theo các tham số của bộ giải mã cũng sẽ tốn thời gian hơn bình thường.
	\item Ở cách ràng buộc trọng số của chúng tôi, khi lan truyền tiến, mặc dù cần phải thực hiện bước chuẩn hóa ở cả bộ mã hóa và bộ giải mã, nhưng nhờ vào ràng buộc $W^{(d)} = (W^{(e)})^T$, ta chỉ cần phải thực hiện bước chuẩn hóa cho bộ trọng số của bộ mã hóa, rồi sau đó dùng lại bộ trọng số đã được chuẩn hóa này cho bộ giải mã. Thời gian huấn luyện của cách ràng buộc này lâu hơn cách ràng buộc $W^{(d)}$ được chuẩn hóa ở trên vì khi lan truyền ngược, ngoài việc tính toán các đạo hàm riêng theo các tham số của bộ giải mã đã được chuẩn hóa, ta cũng cần phải tính toán các đạo hàm riêng theo các tham số của bộ mã hóa đã được chuẩn hóa (khi bộ mã hóa hay bộ giải mã phải thực hiện bước chuẩn hóa khi lan truyền tiến thì việc tính toán các đạo hàm riêng theo các tham số của chúng khi lan truyền ngược sẽ lâu hơn so với khi không thực hiện bước chuẩn hóa).
	\item Ràng buộc $W^{(e)}$ và $W^{(d)}$ được chuẩn hóa có thời gian huấn luyện lâu nhất vì khi lan truyền tiến, ta phải thực hiện bước chuẩn hóa riêng cho bộ mã hóa và bộ giải mã; và khi lan truyền ngược, ta phải tính toán các đạo hàm riêng theo các tham số của bộ giải mã và bộ mã hóa đã được chuẩn hóa.
\end{itemize}
Mặc dù thời gian huấn luyện (trên một vòng lặp) của cách ràng buộc trọng số của chúng tôi là khá cao khi so sánh với cách ràng buộc trọng số khác, nhưng nhìn chung nó vẫn nhanh (nhờ vào việc sử dụng GPU để tính toán song song). Tổng thời gian huấn luyện là khoảng 2.5 giờ.
\begin{table}
	\centering
	\caption[So sánh giữa cách ràng buộc trọng số của chúng tôi với các cách ràng buộc trọng số khác]{So sánh giữa cách ràng buộc trọng số cho SRAEs của chúng tôi với các cách ràng buộc trọng số khác mà có thể áp dụng cho SRAEs. Cách ràng buộc trọng số của chúng tôi giúp SRAEs học được những đặc trưng mà cho kết quả phân lớp (với ``Softmax Regression'') tốt nhất trên tập kiểm tra. Ngoài ra, thời gian huấn luyện trên một vòng lặp của SRAEs với các cách ràng buộc trọng số khác nhau cũng được trình bày ở cột cuối cùng của bảng.}
	\label{table_OurWConstraintVSOtherWConstraints}
	\begin{tabular}{|c|c|c|} \hline
	\textbf{Cách ràng buộc trọng số} & \textbf{\pbox{20cm}{Độ lỗi phân lớp \\trên tập kiểm tra (\%)}} & \textbf{\pbox{20cm}{Thời gian huấn luyện \\của một vòng lặp (giây)}}\\ \hline\hline
	$W^{(d)}$ được chuẩn hóa & 3.28 & 3\\ \hline
	$W^{(e)}$ \& $W^{(d)}$ được chuẩn hóa & 2.51 & 5\\ \hline
	$W^{(d)} = (W^{(e)})^T$ & 2.04 & 2\\ \hline
	Cách ràng buộc của chúng tôi & \textbf{1.62} & 4\\ \hline
	\end{tabular}
\end{table}
\section{SRAEs và các loại ``Auto-Encoders'' khác}
Cuối cùng, chúng tôi cũng so sánh SRAEs (sử dụng cách ràng buộc trọng số của chúng tôi và dùng SW-SGD để huấn luyện) với các loại ``Auto-Encoders'' khác, bao gồm:
\begin{itemize}
	\item \textbf{``Denoising Auto-Encoders'' (DAEs)} \cite{vincent2008extracting}: DAEs muốn học được các đặc trưng ``bền vững'' bằng cách làm nhiễu véc-tơ đầu vào rồi sau đó cố gắng tái tạo lại véc-tơ đầu vào ban đầu từ véc-tơ đã bị làm nhiễu này (véc-tơ đầu vào đã bị làm nhiễu $\rightarrow$ véc-tơ đặc trưng $\rightarrow$ cố gắng tái tạo lại véc-tơ đầu vào không bị nhiễu).
	\item \textbf{``Contractive Auto-Encoders'' (CAEs)} \cite{rifai2011contractive}: DAEs muốn học được các đặc trưng thỏa hai tính chất: (i) có thể tái tạo tốt véc-tơ đầu vào ban đầu, và (ii) bất biến đối với sự thay đổi nhỏ của véc-tơ đầu vào (bằng cách phạt chuẩn Frobenius của ma trận Jacobian của véc-tơ đặc trưng đối với véc-tơ đầu vào).
	\item \textbf{``Higher Order Contractive Auto-Encoders'' (HCAEs)} \cite{rifai2011HCAEs}: HCAEs là mở rộng của CAEs; bên cạnh độ lỗi tái tạo và chuẩn Frobenius của ma trận Jacobian, HCAEs còn phạt thêm chuẩn Frobenius của ma trận Hessian.
\end{itemize}

Bảng \ref{table_SRAEsVSOtherAEs} so sánh các đặc trưng học được (theo độ lỗi phân lớp trên tập kiểm tra) của SRAEs với các loại ``Auto-Encoders'' trên. Với DAEs, CAEs, HCAEs, \cite{rifai2011HCAEs} dùng 1000 nơ-ron ẩn, hàm kích hoạt sigmoid ở cả tầng ẩn và tầng đầu ra, độ lỗi tái tạo ``cross-entropy'', và ràng buộc $W^{(e)}$ và $W^{(d)}$ là chuyển vị của nhau. Như có thể thấy, các đặc trưng học được bởi SRAEs cho kết quả phân lớp (với ``Softmax Regression'') trên tập kiểm tra tốt hơn DAEs và CAEs, nhưng không tốt bằng HCAEs. Tuy nhiên, để ý là HCAEs phức tạp hơn nhiều so với SRAEs của chúng tôi với rất nhiều siêu tham số cần phải lựa chọn.
\begin{table}
	\centering
	\caption[So sánh giữa SRAEs với các loại ``Auto-Encoders'' khác]{So sánh giữa SRAEs (sử dụng cách ràng buộc trọng số của chúng tôi và dùng SW-SGD để huấn luyện) với các loại ``Auto-Encoders'' khác, bao gồm: ``Denoising Auto-Encoders'' (DAEs), ``Contractive Auto-Encoders'' (CAEs), ``Higher Order Contractive Auto-Encoders'' (HCAEs).}
	\label{table_SRAEsVSOtherAEs}
	\begin{tabular}{|c|c|} \hline
		\textbf{Thuật toán học đặc trưng} & \textbf{Độ lỗi phân lớp trên tập kiểm tra (\%)}\\ \hline\hline
		DAEs \cite{rifai2011HCAEs} & 2.05\\ \hline
		CAEs \cite{rifai2011HCAEs} & 1.82\\ \hline
		SRAEs & 1.62\\ \hline
		HCAEs \cite{rifai2011HCAEs} & \textbf{1.20}\\ \hline
	\end{tabular}
\end{table}
	\chapter{Kết Luận Và Hướng Phát Triển}
\ifpdf
    \graphicspath{{Chapter5/Chapter5Figs/PNG/}{Chapter5/Chapter5Figs/PDF/}{Chapter5/Chapter5Figs/}}
\else
    \graphicspath{{Chapter5/Chapter5Figs/EPS/}{Chapter5/Chapter5Figs/}}
\fi
\label{chap_5}

\section{Kết luận}

Trong khóa luận này, chúng tôi nghiên cứu bài toán Dịch máy nơ-ron bằng mô hình Attention-LSTM cộng với 


%\section{Kết chương}


	\def\baselinestretch{1}
\chapter{Kết Luận và Hướng Phát Triển}
\ifpdf
    \graphicspath{{Conclusions/ConclusionsFigs/PNG/}{Conclusions/ConclusionsFigs/PDF/}{Conclusions/ConclusionsFigs/}}
\else
    \graphicspath{{Conclusions/ConclusionsFigs/EPS/}{Conclusions/ConclusionsFigs/}}
\fi

\def\baselinestretch{1.66}

\section{Kết luận}
Trong luận văn này, chúng tôi nghiên cứu về bài toán học đặc trưng không giám sát bằng ``Sparse Auto-Encoders'' (SAEs). SAEs có thể học được những đặc trưng tương tự như ``Sparse Coding'', nhưng điểm lợi là quá trình huấn luyện SAEs có thể được thực hiện một cách hiệu quả thông qua thuật toán lan truyền ngược, và với một véc-tơ đầu vào mới, SAEs có thể tính được véc-tơ đặc trưng tương ứng rất nhanh. Tuy nhiên, trong thực tế, không dễ để có thể làm SAEs ``hoạt động''; có hai điểm ta cần phải làm rõ: (i) ràng buộc thưa, và (ii) ràng buộc trọng số. Đóng góp của luận văn là làm rõ SAEs ở hai điểm này. Cụ thể như sau:
\begin{itemize}
	\item Về ràng buộc thưa, mặc dù chuẩn L1 là cách tự nhiên (vì L1 được dùng trong Sparse Coding) và đơn giản để ràng buộc tính thưa của véc-tơ đặc trưng, nhưng L1 lại thường không được dùng trong SAEs với lý do vẫn còn chưa rõ ràng. Thay vì dùng L1, các bài báo về SAEs thường ràng buộc thưa bằng cách ép giá trị đầu ra trung bình của mỗi nơ-ron ẩn về một giá trị cố định gần $0$. Nhưng giá trị cố định này lại thêm một siêu tham số vào danh sách các siêu tham số vốn đã có rất nhiều của SAEs; điều này sẽ làm cho quá trình chọn lựa các siêu tham số trở nên ``phiền phức'' hơn và tốn thời gian hơn. \emph{Trong luận văn, chúng tôi cố gắng hiểu khó khăn gặp phải khi huấn luyện SAEs với chuẩn L1; từ đó, đề xuất một phiên bản hiệu chỉnh của thuật toán ``Stochastic Gradient Descent'' (SGD), gọi là ``Sleep-Wake Stochastic Gradient Descent'' (SW-SGD), để khắc phục khó khăn gặp phải này. Ở đây, chúng tôi tập trung nghiên cứu SAEs với hàm kích hoạt ``rectified linear'' ở tầng ẩn vì hàm này tính nhanh và có thể cho tính thưa thật sự (đúng bằng 0); chúng tôi gọi SAEs với hàm kích hoạt này là ``Sparse Rectified Auto-Encoders'' (SRAEs).}
	\item Về ràng buộc trọng số, có một số cách đã được đề xuất để ràng buộc trọng số của SAEs, nhưng không rõ là tại sao ta lại nên ràng buộc trọng số như vậy. Liệu có cách ràng buộc trọng số nào tốt hơn? \emph{Trong luận văn, chúng tôi đề xuất một cách ràng buộc trọng số mới và hợp lý cho SRAEs.}
\end{itemize}

Các kết quả thí nghiệm trên bộ dữ liệu MNIST (bộ ảnh chữ số viết tay từ 0 đến 9) cho thấy:
\begin{itemize}
	\item Khi huấn luyện SRAEs với chuẩn L1 sẽ gặp phải vấn đề nơ-ron ``ngủ'' và chiến lược ``ngủ - đánh thức'' đề xuất của chúng tôi trong thuật toán SW-SGD có thể giúp khắc phục vấn đề này.
	\item Cách ràng buộc trọng số đề xuất của chúng tôi giúp SRAEs học được những đặc trưng cho kết quả phân lớp tốt nhất so với các cách ràng buộc trọng số khác mà có thể áp dụng cho SRAEs.
	\item SRAEs với SW-SGD và cách ràng buộc trọng số của chúng tôi có thể học được những đặc trưng cho kết quả phân lớp tốt so với các loại ``Auto-Encoders'' khác.
\end{itemize}
\section{Hướng phát triển} 
Thật ra, luận văn mới chỉ giải quyết được một phần nhỏ và mang tính kỹ thuật (làm cho SAEs hoạt động) của bài toán học đặc trưng không giám sát. Câu hỏi lớn và mang tính định hướng dài hạn là: \emph{Thế nào là một biểu diễn đặc trưng tốt?} Theo GS. Yoshua Bengio, một trong những nhà nghiên cứu tiên phong trong lĩnh vực học biểu diễn đặc trưng, thì: \emph{Một biễu diễn đặc trưng tốt cần \textbf{phân tách (disentangle)} được các yếu tố giải thích ẩn bên dưới}. Để phân tách được các yếu tố giải thích ẩn, ta cần có sự hiểu biết trước (prior) về các yếu tố ẩn. Ở đây, ta quan tâm đến các sự hiểu biết trước mang tính tổng quát, có thể áp dụng để học đặc trưng trong nhiều bài toán liên quan đến trí tuệ nhân tạo (thị giác máy tính, xử lý ngôn ngữ tự nhiên, ...). Định hướng phát triển của luận văn là tích hợp thêm các hiểu biết trước khác vào SAEs nhằm phân tách tốt hơn các yếu tố giải thích ẩn. Dưới đây là một số hiểu biết trước mà có thể tích hợp vào SAEs:
\begin{itemize}
	\item \textbf{Học sâu}: thế giới xung quanh ta có thể được mô tả bằng một kiến trúc phân cấp; cụ thể là, các yếu tố hay các khái niệm (concept) trừu tượng (ví dụ như con mèo, cái cây, ...) bao gồm các khái niệm ít trừu tượng hơn; các khái niệm ít trừu tượng hơn này lại bao gồm các khái niệm ít trừu tượng hơn nữa ... Do đó, ta muốn học nhiều tầng biễu diễn đặc trưng với độ trừu tượng tăng dần. Mặc dù, SRAEs có thể được dùng để học từng tầng đặc trưng một, nhưng mục tiêu mà chúng tôi hướng đến là: học \emph{đồng thời} nhiều tầng biểu diễn đặc trưng một cách không giám sát.
	\item \textbf{Gom cụm tự nhiên}: các mẫu thuộc các lớp khác nhau nằm trên các đa tạp (manifold) khác nhau và các đa tạp này được phân tách tốt với nhau bởi các vùng có mật độ thấp; hơn nữa, số chiều của các đa tạp này nhỏ hơn rất nhiều so với số chiều của không gian ban đầu. Ta thấy rằng sự gom cụm tự nhiên này sẽ dẫn đến tính thưa. Cụ thể là, các đa tạp khác nhau (ứng với các lớp khác nhau) sẽ được mô tả bởi các hệ trục tọa độ khác nhau. Với một véc-tơ đầu vào $x$ thì chỉ có hệ trục tọa độ của đa tạp ứng với lớp mà $x$ thuộc về được kích hoạt. Nếu ta hiểu véc-tơ đặc trưng $h$ của $x$ chứa các hệ số của các hệ trục tọa độ này thì $h$ sẽ thưa bởi vì chỉ có các hệ số của hệ trục tọa độ được kích hoạt là có giá trị khác 0. Do đó, thay vì ràng buộc tính thưa một cách đơn thuần bằng chuẩn L1, ta có thể tìm cách để ràng buộc tính thưa từ góc nhìn gom cụm tự nhiên nói trên.
\end{itemize} 
%%% ----------------------------------------------------------------------

% ------------------------------------------------------------------------

%%% Local Variables: 
%%% mode: latex
%%% TeX-master: "../thesis"
%%% End: 

	\newpage
\chapter*{Phụ Lục: Các Công Trình Đã Công Bố}
\addcontentsline{toc}{chapter}{Phụ Lục: Các Công Trình Đã Công Bố}
\textbf{Hội nghị quốc tế}:
\begin{itemize}
	\item \textbf{K. Tran} and B. Le, ``Demystifying Sparse Rectified Auto-Encoders,'' in \emph{Proceedings of the Fourth Symposium on Information and Communication Technology}, ser. SoICT'13. New York, NY, USA: ACM, 2013, pp. 101–107. [Online]. Available: http://doi.acm.org/10.1145/2542050.2542065 
\end{itemize}
\includepdf[pages=1-3]{docs/SoICT}
\includepdf[pages=1-7]{docs/MyPaper}

	%\input{Introduction}
	%
	%\input{RelatedWork}
	%
	%\input{Overview}
	%
	%\input{Implementation}
	%
	%\input{Modeling}
	%
	%\input{ExperimentalResults}
	%
	%\input{ApplicationAndFutureWork}
	%
	%\input{Conclusion}
	
	\renewcommand{\bibname}{
		\addcontentsline{toc}{chapter}{TÀI LIỆU THAM KHẢO}
		TÀI LIỆU THAM KHẢO
	}
	\bibliographystyle{Classes/IEEEtranS}
	\bibliographystyle{unsrt}
	\bibliography{References/my_bib}
	
	
	%\input{Appendix}
	
\end{document}